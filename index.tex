% Options for packages loaded elsewhere
% Options for packages loaded elsewhere
\PassOptionsToPackage{unicode}{hyperref}
\PassOptionsToPackage{hyphens}{url}
\PassOptionsToPackage{dvipsnames,svgnames,x11names}{xcolor}
%
\documentclass[
  letterpaper,
]{report}
\usepackage{xcolor}
\usepackage[margin=1in]{geometry}
\usepackage{amsmath,amssymb}
\setcounter{secnumdepth}{-\maxdimen} % remove section numbering
\usepackage{iftex}
\ifPDFTeX
  \usepackage[T1]{fontenc}
  \usepackage[utf8]{inputenc}
  \usepackage{textcomp} % provide euro and other symbols
\else % if luatex or xetex
  \usepackage{unicode-math} % this also loads fontspec
  \defaultfontfeatures{Scale=MatchLowercase}
  \defaultfontfeatures[\rmfamily]{Ligatures=TeX,Scale=1}
\fi
\usepackage{lmodern}
\ifPDFTeX\else
  % xetex/luatex font selection
\fi
% Use upquote if available, for straight quotes in verbatim environments
\IfFileExists{upquote.sty}{\usepackage{upquote}}{}
\IfFileExists{microtype.sty}{% use microtype if available
  \usepackage[]{microtype}
  \UseMicrotypeSet[protrusion]{basicmath} % disable protrusion for tt fonts
}{}
\makeatletter
\@ifundefined{KOMAClassName}{% if non-KOMA class
  \IfFileExists{parskip.sty}{%
    \usepackage{parskip}
  }{% else
    \setlength{\parindent}{0pt}
    \setlength{\parskip}{6pt plus 2pt minus 1pt}}
}{% if KOMA class
  \KOMAoptions{parskip=half}}
\makeatother
% Make \paragraph and \subparagraph free-standing
\makeatletter
\ifx\paragraph\undefined\else
  \let\oldparagraph\paragraph
  \renewcommand{\paragraph}{
    \@ifstar
      \xxxParagraphStar
      \xxxParagraphNoStar
  }
  \newcommand{\xxxParagraphStar}[1]{\oldparagraph*{#1}\mbox{}}
  \newcommand{\xxxParagraphNoStar}[1]{\oldparagraph{#1}\mbox{}}
\fi
\ifx\subparagraph\undefined\else
  \let\oldsubparagraph\subparagraph
  \renewcommand{\subparagraph}{
    \@ifstar
      \xxxSubParagraphStar
      \xxxSubParagraphNoStar
  }
  \newcommand{\xxxSubParagraphStar}[1]{\oldsubparagraph*{#1}\mbox{}}
  \newcommand{\xxxSubParagraphNoStar}[1]{\oldsubparagraph{#1}\mbox{}}
\fi
\makeatother


\usepackage{longtable,booktabs,array}
\usepackage{calc} % for calculating minipage widths
% Correct order of tables after \paragraph or \subparagraph
\usepackage{etoolbox}
\makeatletter
\patchcmd\longtable{\par}{\if@noskipsec\mbox{}\fi\par}{}{}
\makeatother
% Allow footnotes in longtable head/foot
\IfFileExists{footnotehyper.sty}{\usepackage{footnotehyper}}{\usepackage{footnote}}
\makesavenoteenv{longtable}
\usepackage{graphicx}
\makeatletter
\newsavebox\pandoc@box
\newcommand*\pandocbounded[1]{% scales image to fit in text height/width
  \sbox\pandoc@box{#1}%
  \Gscale@div\@tempa{\textheight}{\dimexpr\ht\pandoc@box+\dp\pandoc@box\relax}%
  \Gscale@div\@tempb{\linewidth}{\wd\pandoc@box}%
  \ifdim\@tempb\p@<\@tempa\p@\let\@tempa\@tempb\fi% select the smaller of both
  \ifdim\@tempa\p@<\p@\scalebox{\@tempa}{\usebox\pandoc@box}%
  \else\usebox{\pandoc@box}%
  \fi%
}
% Set default figure placement to htbp
\def\fps@figure{htbp}
\makeatother





\setlength{\emergencystretch}{3em} % prevent overfull lines

\providecommand{\tightlist}{%
  \setlength{\itemsep}{0pt}\setlength{\parskip}{0pt}}



 


\usepackage{booktabs}
\usepackage{longtable}
\usepackage{array}
\usepackage{multirow}
\usepackage{wrapfig}
\usepackage{float}
\usepackage{colortbl}
\usepackage{pdflscape}
\usepackage{tabu}
\usepackage{threeparttable}
\usepackage{threeparttablex}
\usepackage[normalem]{ulem}
\usepackage{makecell}
\usepackage{xcolor}
\makeatletter
\@ifpackageloaded{tcolorbox}{}{\usepackage[skins,breakable]{tcolorbox}}
\@ifpackageloaded{fontawesome5}{}{\usepackage{fontawesome5}}
\definecolor{quarto-callout-color}{HTML}{909090}
\definecolor{quarto-callout-note-color}{HTML}{0758E5}
\definecolor{quarto-callout-important-color}{HTML}{CC1914}
\definecolor{quarto-callout-warning-color}{HTML}{EB9113}
\definecolor{quarto-callout-tip-color}{HTML}{00A047}
\definecolor{quarto-callout-caution-color}{HTML}{FC5300}
\definecolor{quarto-callout-color-frame}{HTML}{acacac}
\definecolor{quarto-callout-note-color-frame}{HTML}{4582ec}
\definecolor{quarto-callout-important-color-frame}{HTML}{d9534f}
\definecolor{quarto-callout-warning-color-frame}{HTML}{f0ad4e}
\definecolor{quarto-callout-tip-color-frame}{HTML}{02b875}
\definecolor{quarto-callout-caution-color-frame}{HTML}{fd7e14}
\makeatother
\makeatletter
\@ifpackageloaded{bookmark}{}{\usepackage{bookmark}}
\makeatother
\makeatletter
\@ifpackageloaded{caption}{}{\usepackage{caption}}
\AtBeginDocument{%
\ifdefined\contentsname
  \renewcommand*\contentsname{Table of contents}
\else
  \newcommand\contentsname{Table of contents}
\fi
\ifdefined\listfigurename
  \renewcommand*\listfigurename{List of Figures}
\else
  \newcommand\listfigurename{List of Figures}
\fi
\ifdefined\listtablename
  \renewcommand*\listtablename{List of Tables}
\else
  \newcommand\listtablename{List of Tables}
\fi
\ifdefined\figurename
  \renewcommand*\figurename{Figure}
\else
  \newcommand\figurename{Figure}
\fi
\ifdefined\tablename
  \renewcommand*\tablename{Table}
\else
  \newcommand\tablename{Table}
\fi
}
\@ifpackageloaded{float}{}{\usepackage{float}}
\floatstyle{ruled}
\@ifundefined{c@chapter}{\newfloat{codelisting}{h}{lop}}{\newfloat{codelisting}{h}{lop}[chapter]}
\floatname{codelisting}{Listing}
\newcommand*\listoflistings{\listof{codelisting}{List of Listings}}
\makeatother
\makeatletter
\makeatother
\makeatletter
\@ifpackageloaded{caption}{}{\usepackage{caption}}
\@ifpackageloaded{subcaption}{}{\usepackage{subcaption}}
\makeatother
\makeatletter
\@ifpackageloaded{sidenotes}{}{\usepackage{sidenotes}}
\@ifpackageloaded{marginnote}{}{\usepackage{marginnote}}
\makeatother
\usepackage{bookmark}
\IfFileExists{xurl.sty}{\usepackage{xurl}}{} % add URL line breaks if available
\urlstyle{same}
\hypersetup{
  pdftitle={Property Development Opportunity Analysis},
  pdfauthor={HDAdvisors},
  colorlinks=true,
  linkcolor={blue},
  filecolor={Maroon},
  citecolor={Blue},
  urlcolor={Blue},
  pdfcreator={LaTeX via pandoc}}


\title{Property Development Opportunity Analysis}
\usepackage{etoolbox}
\makeatletter
\providecommand{\subtitle}[1]{% add subtitle to \maketitle
  \apptocmd{\@title}{\par {\large #1 \par}}{}{}
}
\makeatother
\subtitle{A Comprehensive Assessment of Underutilized Assets}
\author{HDAdvisors}
\date{2025-12-17}
\begin{document}
\maketitle

\renewcommand*\contentsname{Table of contents}
{
\hypersetup{linkcolor=}
\setcounter{tocdepth}{2}
\tableofcontents
}

\bookmarksetup{startatroot}

\chapter{Property Development Opportunity
Analysis}\label{property-development-opportunity-analysis}

\section{Report Purpose}\label{report-purpose}

This analysis, done for the Virginia Episcopal Real Estate Portfolio
(VEREP), provides:

\begin{itemize}
\tightlist
\item
  \textbf{Strategic Insights}: Identification of high-potential
  development opportunities amongst their smallest congregations
\item
  \textbf{Data-Driven Decisions}: Evidence-based property assessments
\item
  \textbf{Geographic Analysis}: Regional distribution and market
  potential
\end{itemize}

\section{How to Use This Report}\label{how-to-use-this-report}

\begin{itemize}
\tightlist
\item
  \textbf{Executive Summary}: Quick project overview and key findings
\item
  \textbf{Methodology}: Understanding the analytical approach to
  assessing development potential
\item
  \textbf{Analysis Results}: Detailed findings and insights of relevant
  parcels
\item
  \textbf{Summary Statistics}: Portfolio-wide metrics
\item
  \textbf{Property Profiles}: Individual property assessments for top
  identified parcels
\item
  \textbf{Appendices}: Technical details and references
\end{itemize}

\begin{tcolorbox}[enhanced jigsaw, breakable, bottomrule=.15mm, title=\textcolor{quarto-callout-tip-color}{\faLightbulb}\hspace{0.5em}{Interactive Features}, toprule=.15mm, opacitybacktitle=0.6, toptitle=1mm, left=2mm, bottomtitle=1mm, coltitle=black, titlerule=0mm, colbacktitle=quarto-callout-tip-color!10!white, colback=white, leftrule=.75mm, arc=.35mm, colframe=quarto-callout-tip-color-frame, rightrule=.15mm, opacityback=0]

This report includes interactive visualizations and tools. Use the
filters and controls to explore the data dynamically.

\end{tcolorbox}

\section{Navigation}\label{navigation}

Use the table of contents on the left to navigate between sections, or
use the search function to find specific topics.

\bookmarksetup{startatroot}

\chapter*{Executive Summary}\label{executive-summary}
\addcontentsline{toc}{chapter}{Executive Summary}

\markboth{Executive Summary}{Executive Summary}

This analysis examines properties in the Episcopal Diocese of Virginia
with fewer than 30 Sunday attendees, focusing on development potential
through key metrics outlined in the analytical framework chapter. This
report was done for VEREP (Virginia Episcopal Real Estate Partners) in
conjunction with case study projects HDAdvisors is facilitating.

\section*{At a Glance: Top 10
Parcelss}\label{at-a-glance-top-10-parcelss}
\addcontentsline{toc}{section}{At a Glance: Top 10 Parcelss}

\markright{At a Glance: Top 10 Parcelss}

\begin{verbatim}
✓ Scoring complete

=== TIER DISTRIBUTION ===
# A tibble: 5 x 2
  development_tier               n
  <chr>                      <int>
1 Tier 4: Limited Potential     37
2 Tier 5: Not Recommended       37
3 Tier 3: Moderate Potential    24
4 Tier 2: Strong Potential       4
5 Tier 1: High Priority          2

=== GOLDILOCKS OPPORTUNITIES ===
Total High + Moderate: 39 (37.5%)

=== TOP 15 PROPERTIES BY SCORE ===
# A tibble: 15 x 7
    rank congregation_name scity area_acres development_score constraint_summary
   <int> <chr>             <chr>      <dbl>             <dbl> <chr>             
 1     1 St Pauls on the ~ WINC~      3.20               79.2 None              
 2     2 St Pauls on the ~ WINC~      1.67               76   None              
 3     3 Christ Ascension~ RICH~      1.47               64.8 None              
 4     4 Good Shepherd-of~ FREE~      3.12               63.2 None              
 5     5 St Johns Church   COLU~      0.516              62   None              
 6     6 Grace Church Emm~ MIDL~      3.65               58.2 None              
 7     7 Grace Church Emm~ MIDL~      2.76               58.2 None              
 8     8 St Georges Church STAN~      1.83               57   None              
 9     9 Aquia Church      STAF~      1.12               54.6 None              
10    10 Calvary Episcopa~ HANO~      1.45               54.3 None              
11    11 Holy Cross Korea~ FAIR~      3.44               52.8 Historic District 
12    12 Aquia Church      STAF~      1.53               50.8 Historic District 
13    13 Aquia Church      STAF~      0.974              50.6 None              
14    14 Aquia Church      STAF~      0.948              50.6 None              
15    15 All Saints Sharo~ ALEX~      4.39               50.2 None              
# i 1 more variable: development_potential <chr>

=== CONSTRAINT IMPACT ===
# A tibble: 1 x 5
  total with_constraints avg_score_unconstrained avg_score_constrained
  <int>            <int>                   <dbl>                 <dbl>
1    39               22                    56.4                  34.9
# i 1 more variable: score_difference <dbl>

=== CONSTRAINT TYPES IN GOLDILOCKS ===
# A tibble: 4 x 2
  constraint_summary             n
  <chr>                      <int>
1 None                          17
2 Historic District             15
3 Moderate Hazard Risk           6
4 Historic + Moderate Hazard     1

✓ Files saved:
  - property_profile_scored.rds
  - property_profile_scored.csv
  - verep_top_opportunities.csv
\end{verbatim}

\begin{longtable}[t]{>{\centering\arraybackslash}p{3em}llllr>{}rll}

\caption{\label{tbl-top-dev}Top 10 Development Opportunities}

\tabularnewline

\toprule
Rank & Congregation & Address & City & County & Acres & Score & Constraints & Tier\\
\midrule
\textbf{1} & St Pauls on the Hill Church & 1527 SENSENY RD, WINCHESTER 22602 & WINCHESTER & FREDERICK & 3.20 & \textcolor[HTML]{8baeaa}{\textbf{79.2}} & None & Tier 1: High Priority\\
\textbf{2} & St Pauls on the Hill Church & WINCHESTER 22602 & WINCHESTER & FREDERICK & 1.67 & \textcolor[HTML]{8baeaa}{\textbf{76.0}} & None & Tier 1: High Priority\\
\textbf{3} & Christ Ascension Church & 1704 WEST LABURNUM AVE, RICHMOND 23227 & RICHMOND & RICHMOND-CITY & 1.47 & \textcolor[HTML]{e9ab3f}{\textbf{64.8}} & None & Tier 2: Strong Potential\\
\textbf{4} & Good Shepherd-of-the-Hills & 7420 MISSION HOME RD, FREE UNION 22940 & FREE UNION & ALBEMARLE & 3.12 & \textcolor[HTML]{e9ab3f}{\textbf{63.2}} & None & Tier 2: Strong Potential\\
\textbf{5} & St Johns Church & 64 K ST, COLUMBIA 23038 & COLUMBIA & FLUVANNA & 0.52 & \textcolor[HTML]{e9ab3f}{\textbf{62.0}} & None & Tier 2: Strong Potential\\
\addlinespace
\textbf{6} & Grace Church Emmanuel Parish & 5096 GRACE CHURCH LN, MIDLAND 22728 & MIDLAND & FAUQUIER & 3.65 & \textcolor[HTML]{e9ab3f}{\textbf{58.2}} & None & Tier 3: Moderate Potential\\
\textbf{7} & Grace Church Emmanuel Parish & 5094 GRACE CHURCH LN, MIDLAND 22728 & MIDLAND & FAUQUIER & 2.76 & \textcolor[HTML]{e9ab3f}{\textbf{58.2}} & None & Tier 3: Moderate Potential\\
\textbf{8} & St Georges Church & 3392 PINE GROVE RD, STANLEY 22851 & STANLEY & PAGE & 1.83 & \textcolor[HTML]{e76f52}{\textbf{57.0}} & None & Tier 3: Moderate Potential\\
\textbf{9} & Aquia Church & STAFFORD 22554 & STAFFORD & STAFFORD & 1.12 & \textcolor[HTML]{e76f52}{\textbf{54.6}} & None & Tier 3: Moderate Potential\\
\textbf{10} & Calvary Episcopal Church & 13312 HANOVER COURTHOUSE RD, HANOVER 23069 & HANOVER & HANOVER & 1.45 & \textcolor[HTML]{e76f52}{\textbf{54.3}} & None & Tier 3: Moderate Potential\\
\bottomrule

\end{longtable}

\section*{Key Findings}\label{key-findings}
\addcontentsline{toc}{section}{Key Findings}

\markright{Key Findings}

\subsection*{Summary Overview}\label{summary-overview}
\addcontentsline{toc}{subsection}{Summary Overview}

\begin{figure}

\begin{minipage}{0.25\linewidth}

\begin{tcolorbox}[enhanced jigsaw, breakable, left=2mm, bottomrule=.15mm, opacityback=0, colback=white, leftrule=.75mm, arc=.35mm, colframe=quarto-callout-note-color-frame, rightrule=.15mm, toprule=.15mm]

\vspace{-3mm}\textbf{100}\vspace{3mm}

Parcels Analyzed

\end{tcolorbox}

\end{minipage}%
%
\begin{minipage}{0.25\linewidth}

\begin{tcolorbox}[enhanced jigsaw, breakable, left=2mm, bottomrule=.15mm, opacityback=0, colback=white, leftrule=.75mm, arc=.35mm, colframe=quarto-callout-warning-color-frame, rightrule=.15mm, toprule=.15mm]

\vspace{-3mm}\textbf{64.1}\vspace{3mm}

Strong Potential Acres

\end{tcolorbox}

\end{minipage}%
%
\begin{minipage}{0.25\linewidth}

\begin{tcolorbox}[enhanced jigsaw, breakable, left=2mm, bottomrule=.15mm, opacityback=0, colback=white, leftrule=.75mm, arc=.35mm, colframe=quarto-callout-important-color-frame, rightrule=.15mm, toprule=.15mm]

\vspace{-3mm}\textbf{248}\vspace{3mm}

Total Acres Analyzed

\end{tcolorbox}

\end{minipage}%

\end{figure}%

\subsection*{Significant Underutilized
Assets}\label{significant-underutilized-assets}
\addcontentsline{toc}{subsection}{Significant Underutilized Assets}

\begin{figure}

\begin{minipage}{0.33\linewidth}

\begin{tcolorbox}[enhanced jigsaw, breakable, left=2mm, bottomrule=.15mm, opacityback=0, colback=white, leftrule=.75mm, arc=.35mm, colframe=quarto-callout-warning-color-frame, rightrule=.15mm, toprule=.15mm]

\vspace{-3mm}\textbf{3 Parcels (3.9 acres)}\vspace{3mm}

Surface Parking Lots

\end{tcolorbox}

\end{minipage}%
%
\begin{minipage}{0.33\linewidth}

\begin{tcolorbox}[enhanced jigsaw, breakable, left=2mm, bottomrule=.15mm, opacityback=0, colback=white, leftrule=.75mm, arc=.35mm, colframe=quarto-callout-warning-color-frame, rightrule=.15mm, toprule=.15mm]

\vspace{-3mm}\textbf{22 Parcels (98.6 acres)}\vspace{3mm}

Vacant/Open Space

\end{tcolorbox}

\end{minipage}%
%
\begin{minipage}{0.33\linewidth}

\begin{tcolorbox}[enhanced jigsaw, breakable, left=2mm, bottomrule=.15mm, opacityback=0, colback=white, leftrule=.75mm, arc=.35mm, colframe=quarto-callout-tip-color-frame, rightrule=.15mm, toprule=.15mm]

\vspace{-3mm}\textbf{56 Parcels (81.6 acres)}\vspace{3mm}

Financial Need Congregation Parcels

\end{tcolorbox}

\end{minipage}%

\end{figure}%

This defines ``financial need'' as:

\begin{itemize}
\tightlist
\item
  Sites with declining congregations
\item
  Sites with declining giving
\end{itemize}

This is further explained in the methodology section
{[}@FinancialNeed{]}.

\subsection*{Geographic Distribution}\label{geographic-distribution}
\addcontentsline{toc}{subsection}{Geographic Distribution}

\begin{figure}[H]

\centering{

\pandocbounded{\includegraphics[keepaspectratio]{executive-summary_files/figure-pdf/fig-geographic-dists-1.pdf}}

}

\caption{\label{fig-geographic-dists}Parcels by County}

\end{figure}%

Development opportunities are distributed across Virginia, with
concentrations in:

\begin{itemize}
\tightlist
\item
  Urban/suburban areas with strong walkability scores
\item
  Markets with favorable demographics and LIHTC eligibility
\item
  Transit-accessible locations supporting reduced parking requirements
\end{itemize}

\subsection*{Congregation Vitality
Summary}\label{congregation-vitality-summary}
\addcontentsline{toc}{subsection}{Congregation Vitality Summary}

\begin{longtable*}[t]{lr}
\toprule
\cellcolor[HTML]{445ca9}{\textcolor{white}{\textbf{Metric}}} & \cellcolor[HTML]{445ca9}{\textcolor{white}{\textbf{Value}}}\\
\midrule
\textbf{Congregations with <30 Sunday Attendance} & \textbf{37}\\
Total Parcels Analyzed & 104\\
Average Parcels per Congregation & 2.8\\
Average Sunday Attendance & 17 people\\
Average Annual Pledge & \$44,271\\
\addlinespace
\textbf{Congregations in Financial Need} & \textbf{21}\\
\bottomrule
\end{longtable*}

\section*{Next Steps}\label{next-steps}
\addcontentsline{toc}{section}{Next Steps}

\markright{Next Steps}

\begin{enumerate}
\def\labelenumi{\arabic{enumi}.}
\tightlist
\item
  \textbf{Priority Assessment}: Review top-ranked properties for
  immediate action
\item
  \textbf{Feasibility Studies}: Conduct detailed feasibility analysis
  for top candidates
\item
  \textbf{Stakeholder Engagement}: Engage relevant stakeholders for
  high-priority sites
\item
  \textbf{Financial Planning}: Develop funding strategies and timelines
\end{enumerate}

\part{Methodology \& Data}

\chapter{Background \& Data Sources}\label{background-data-sources}

\section{VEREP Parcel Dataset}\label{verep-parcel-dataset}

This analysis utilizes the Virginia Episcopal Real Estate Portfolio
(VEREP) parcel dataset, compiled by CGS Consultants. The dataset
(originally 790 parcels) was further filtered to look at
parcels/properties owned by congregations or the diocese and with small
congregations to total:

\begin{itemize}
\item
  \textbf{104 parcels} across Virginia
\item
  Comprehensive property characteristics (size, use, zoning)
\item
  Environmental constraints (floodplains, wetlands, easements)
\item
  Location metrics (walkability, transit access)
\item
  Market indicators (median income, demographics)
\end{itemize}

\section{Congregation Statistics
(2014-2023)}\label{congregation-statistics-2014-2023}

This analysis also includes congregation statistics from the VEREP GIS
webtool. Financial and membership data provides context on congregation
health:

\begin{itemize}
\tightlist
\item
  Sunday attendance trends
\item
  Membership changes
\item
  Plate \& pledge revenue
\item
  10-year trend analysis
\end{itemize}

\begin{tcolorbox}[enhanced jigsaw, breakable, bottomrule=.15mm, title=\textcolor{quarto-callout-note-color}{\faInfo}\hspace{0.5em}{Data Quality Note}, toprule=.15mm, opacitybacktitle=0.6, toptitle=1mm, left=2mm, bottomtitle=1mm, coltitle=black, titlerule=0mm, colbacktitle=quarto-callout-note-color!10!white, colback=white, leftrule=.75mm, arc=.35mm, colframe=quarto-callout-note-color-frame, rightrule=.15mm, opacityback=0]

Not all properties have associated congregation data. Properties without
congregation information received neutral scores in the financial need
category.

\end{tcolorbox}

\section{Data Quality Issues}\label{data-quality-issues}

Of the 104 parcels analyzed, 101 (97.1\%) have incomplete data ---
meaning they are missing at least one of the six key development metrics
(land value, acreage, wetland coverage, flood zone, QCT status, or
zoning). This high rate reflects broader data quality challenges
identified in CGS's initial VEREP study.

While environmental constraint data (wetlands, flood zones) accounts for
much of the missing information, 15 properties (14.4\%) are missing land
value, limiting our ability to assess their financial potential. The
breakdown below classifies these data gaps to help prioritize follow-up
research. Once these are verified, these properties can be more
precisely evaluated for development opportunities.

\pandocbounded{\includegraphics[keepaspectratio]{methodology-overview_files/figure-pdf/missing-data-chart-1.pdf}}

\section{Addressing Data Gaps Through Enhanced
Assessment}\label{addressing-data-gaps-through-enhanced-assessment}

In the interim, we attempted to use other data present to fill in these
gaps. This process is outlined below.

\subsection{The QCT Data Challenge}\label{the-qct-data-challenge}

Qualified Census Tract (QCT) designation applied to zero properties in
this portfolio, rendering it useless for market assessment. We replaced
QCT with median income and five-year household income growth forecasts
from census tract data. Properties in high-income, high-growth areas
score highest (90-95 points), while those in declining markets score
lower (35-40 points). This approach uses actual market conditions at
each property location rather than federal designations that proved
inapplicable to small, rural congregations.

\subsection{The Wetland Data
Challenge}\label{the-wetland-data-challenge}

The most significant data quality issue emerged in environmental
constraint assessment: \textbf{97\% of properties lacked wetland
coverage data} from the National Wetlands Inventory. This critical gap
meant that relying solely on wetland percentages would fail to identify
environmentally constrained properties, potentially overestimating
development opportunities across the portfolio.

Rather than impute wetland data or exclude these properties from
analysis, we implemented a \textbf{multi-faceted constraint assessment
framework} using complementary data sources with comprehensive coverage:

\textbf{1 Historic District Designation}

We incorporated National Register Historic District boundaries from the
Virginia Department of Historic Resources. This revealed that
\textbf{45\% of the 104 properties} (n=47) are located within historic
districts---a finding with significant implications for development
feasibility. Properties in historic districts face additional regulatory
requirements including design review processes, material restrictions,
and extended approval timelines. Historic designation doesn't prohibit
development but typically adds 15-25\% to project costs and 3-6 months
to development schedules. These properties receive a \textbf{-25 point
penalty} in development scoring to reflect added complexity.

\textbf{2 FEMA National Risk Index}

The FEMA National Risk Index provides comprehensive natural hazard
assessment beyond flood zones alone, incorporating risks from
hurricanes, tornadoes, severe storms, wildfires, earthquakes, and other
hazards. This multi-hazard approach captures the full spectrum of
climate and disaster risks that affect development feasibility,
construction costs, and insurance requirements.

Among the 104 properties:

\begin{itemize}
\tightlist
\item
  \textbf{Very Low or Relatively Low risk}: 79\% (minimal concerns)
\item
  \textbf{Relatively Moderate risk}: 32\% (manageable with standard
  precautions, -15 point penalty)
\item
  \textbf{Relatively High risk}: 12\% (significant mitigation required,
  -40 point penalty)
\item
  \textbf{Very High risk}: \textless1\% (likely deal-breaker, -50 point
  penalty)
\end{itemize}

\textbf{Combined Assessment Impact}

This enhanced screening framework identified environmental and
regulatory constraints affecting \textbf{17 properties} (16\%) that
would have been missed using wetland data alone. The addition of
historic district data proved particularly valuable: among the 36
goldilocks opportunities (0.5-5 acres), \textbf{16 properties} (44\%)
carry historic designation, providing crucial context for development
planning and timeline expectations.

For the small subset of properties with wetland data available (11\%),
we retained this information and applied it alongside historic and
hazard assessments. Properties with multiple major constraints (e.g.,
flood zone + high hazard risk, or significant wetlands + flood zone)
receive compounded penalties (up to -75 points), reflecting cumulative
impacts on development feasibility.

\subsection{Addressing Missing Property
Information}\label{addressing-missing-property-information}

Eleven properties lacked specific street addresses in county parcel
records, listed instead as ``UNASSIGNED'' or missing entirely. To enable
spatial analysis and mapping, we employed the Geocodio API to generate
approximate coordinates based on available city and county information.

\textbf{Geocoding Results:}

\begin{itemize}
\tightlist
\item
  \textbf{Success rate}: 100\% (11/11 properties geocoded)
\item
  \textbf{Accuracy}: Medium (0.54) representing city-center
  approximations
\item
  \textbf{Location}: All 11 properties geocoded to West Point, King
  William County
\item
  \textbf{Congregations}: Two congregations affected (St.~Pauls Church,
  St.~Johns Episcopal Church)
\end{itemize}

\textbf{Analysis Impact:}

The geocoding process successfully provided coordinates for mapping
while revealing that all 11 properties are \textbf{small parcels under
0.5 acres}---below the goldilocks development threshold. These
properties were appropriately classified as ``Small Parcel'' regardless
of address precision, meaning the city-center coordinate approximations
do not affect development opportunity identification. All 104 properties
now have sufficient location data for spatial analysis and
visualization, with no impact on the integrity of top opportunities
rankings.

\textbf{Remaining Data Needs:}

Following wetland data supplementation and geocoding remediation,
\textbf{15 properties} (14.4\%) lacked assessed land values from county
records. Rather than excluding these properties or pursuing individual
county assessor verification, which would be more reliable but an added
task, we applied a conservative estimation method using existing census
tract-level housing market data.

\subsubsection{Land Value Estimation
Method:}\label{land-value-estimation-method}

Properties without assessed land values were estimated using census
tract median home values, assuming land represents approximately 25\% of
total residential property value in the area. This ratio aligns with
typical land-to-improvement value relationships in suburban and rural
Virginia markets. The formula applied:

\begin{verbatim}
Estimated Land Value = (Census Tract Median Home Value × 0.25) × Property Acreage
\end{verbatim}

This method provides order-of-magnitude values sufficient for
preliminary opportunity screening while flagging properties for detailed
appraisal before advancing to feasibility analysis. All estimated values
are clearly marked in the dataset with the source noted as ``Estimated
from Area Median'' rather than ``County Assessor.''

\textbf{Coverage Achievement:}

\begin{itemize}
\tightlist
\item
  Properties with direct assessor land values: 89/104 (85.6\%)
\item
  Properties with estimated land values: 15/104 (14.4\%)
\item
  Total properties with usable land value data: \textbf{104/104 (100\%)}
\end{itemize}

No properties require manual follow-up for land value data, though the
15 estimated values should be verified through formal appraisal during
due diligence for any properties advancing to development.

\subsection{Before and after enhanced
assessment}\label{before-and-after-enhanced-assessment}

\pandocbounded{\includegraphics[keepaspectratio]{methodology-overview_files/figure-pdf/unnamed-chunk-2-1.pdf}}

\begin{longtable}[t]{lr>{}r>{}rll}
\caption{\label{tab:unnamed-chunk-3}Data Enhancement Impact Summary}\\
\toprule
Data Category & Original Gap & Addressed & Remaining & Enhancement & Explanation\\
\midrule
Market Indicators & 104 & \textcolor[HTML]{8baeaa}{\textbf{104}} & \textcolor[HTML]{666666}{0} & Median income + 5-year household income growth forecasts & Replaced ineffective QCT designation (0 properties qualified) with actual market conditions\\
Wetland Data & 97 & \textcolor[HTML]{8baeaa}{\textbf{97}} & \textcolor[HTML]{666666}{0} & Historic districts + FEMA National Risk Index & Multi-faceted constraint framework capturing regulatory and hazard risks\\
Land Value & 15 & \textcolor[HTML]{8baeaa}{\textbf{15}} & \textcolor[HTML]{666666}{0} & Census tract median home value estimation & Conservative estimation assuming land = 25\% of residential property value\\
Property Address & 11 & \textcolor[HTML]{8baeaa}{\textbf{11}} & \textcolor[HTML]{666666}{0} & Geocodio API geocoding & City-center approximations sufficient for small parcels excluded from top opportunities\\
\bottomrule
\end{longtable}

\subsection{Recommendation}\label{recommendation}

The enhanced constraint assessment framework using historic districts
and comprehensive hazard ratings provides robust environmental and
regulatory screening despite limited wetland data. Properties identified
as opportunities have been vetted against multiple constraint sources,
providing confidence in their development potential. Future
site-specific due diligence should include wetland delineation for
properties advancing to development, particularly for the 89\% of
parcels without existing wetland assessments.

\chapter{Data Source Details}\label{data-source-details}

The parcel data used for this analysis originates from the Center for
Geospatial Solutions (CGS) WHOA parcel ownership dataset, which combines
cleaned and standardized parcel data from Regrid, property records from
ATTOM, and corporate entity registration data from OpenCorporates into a
unified national dataset.

CGS identified parcels of interest to VEREP through a multi-step
process: geocoding addresses provided by the Episcopal Diocese of
Virginia, cross-referencing owner names and mailing addresses against
Virginia parcel records, and applying keyword searches specific to
Episcopal-affiliated entities. The methodology included manual
verification to remove parcels belonging to other dioceses or
denominations (such as the Anglican Church in North America or African
Methodist Episcopal Church) and to reconcile 572 stacked parcels into 44
distinct records.

The final CGS dataset, completed in February 2025, contains 798 parcels
totaling approximately 3,707 acres with an assessed value exceeding
\$1.79 billion. While the majority of this data was not utilized for
this particular assessment, similar development analysis could be
performed in the future with the complete parcel data.

\chapter{Analytical Framework}\label{analytical-framework}

\section{Development Potential
Scoring}\label{development-potential-scoring}

Our methodology evaluates \textbf{seven key criteria} across positive
attributes and limiting factors to create a composite development score
(0-100):

\subsection{Six Weighted Components (total:
100\%)}\label{six-weighted-components-total-100}

\pandocbounded{\includegraphics[keepaspectratio]{analytical-framework_files/figure-pdf/scoring-visual-1.pdf}}

\subsection{Plus Constraint Penalties}\label{plus-constraint-penalties}

Environmental and regulatory factors are applied as score deductions:

\begin{itemize}
\tightlist
\item
  Combined flood zone + significant wetlands (\textgreater25\%):
  \textbf{-75 points}
\item
  Flood zone OR moderate wetlands (\textgreater10\%): \textbf{-40
  points}\\
\item
  Minor environmental constraints: \textbf{-20 points}
\end{itemize}

\subsection{Development Classification
Systems}\label{development-classification-systems}

Properties are evaluated using two complementary frameworks:

\textbf{Physical Classification} (for opportunity identification):

\begin{itemize}
\tightlist
\item
  \textbf{High Priority}: 1.5-5 acres, minimal constraints, documented
  value
\item
  \textbf{Moderate Priority}: 0.5-1.5 acres, minimal constraints
\item
  \textbf{Too Large}: \textgreater5 acres (may require phased
  development or specialized partners)
\item
  \textbf{Small Parcel}: \textless0.5 acres (limited development
  options)
\item
  \textbf{Constrained}: Environmental limitations regardless of size
\end{itemize}

\textbf{Numerical Score Tiers} (for prioritization):

\begin{itemize}
\tightlist
\item
  \textbf{Tier 1 (75-100):} High Priority - Immediate development
  candidates
\item
  \textbf{Tier 2 (60-74):} Strong Potential - Near-term opportunities
\item
  \textbf{Tier 3 (45-59):} Moderate Potential - Requires creative
  solutions
\item
  \textbf{Tier 4 (30-44):} Limited Potential - Significant barriers
  exist
\item
  \textbf{Tier 5 (\textless30):} Not Recommended - Unsuitable for
  development
\end{itemize}

\subsection{Why use BOTH?}\label{why-use-both}

Consider two hypothetical properties:

\textbf{St.~Michael's Church, Suburbia County} - 2.1 acres, no
constraints, assessed at \$450K - Congregation: 45 members, stable
finances, no urgency to act - Size Classification: \textbf{High
Priority} (meets all physical criteria) - Development Score: \textbf{52}
(Tier 3: Moderate Potential)

This property \emph{can} be developed---it passes the physical
checklist---but \emph{should} it be prioritized? The congregation isn't
financially stressed, the market is lukewarm, and leadership has shown
no interest in partnerships. The score reflects this: physically viable,
strategically underwhelming.

\textbf{Grace Memorial Church, Growth County} - 1.3 acres, no
constraints, assessed at \$380K - Congregation: 12 members, 60\% decline
in giving, aging building - Size Classification: \textbf{Moderate
Priority} (just under 1.5 acre threshold) - Development Score:
\textbf{71} (Tier 2: Strong Potential)

This property narrowly misses the ``High Priority'' physical threshold,
but the score tells a different story: a struggling congregation in a
strong market, likely motivated to explore options. The 0.2-acre
shortfall matters less than the strategic opportunity.

Using physical classification alone, St.~Michael's looks like the better
opportunity. Using score alone, Grace Memorial wins. The truth requires
both lenses:

\begin{itemize}
\tightlist
\item
  Size classification answers: \emph{``What's physically possible?''}
\item
  Score tiers answer: \emph{``What's strategically worthwhile?''}
\end{itemize}

The ideal candidates---like your top-ranked properties---excel on both
dimensions. But when they diverge, the score typically better predicts
which conversations will be productive and which properties will
actually move toward development. While this process can only truly be
perfected by those with congregational history and knowledge, this is
the assessment's attempt to find the best of both worlds.

The physical classification criteria is discussed in more detail in the
next section of this report.

\chapter{Understanding the Development
Criteria}\label{understanding-the-development-criteria}

\section{Why These Factors Matter}\label{why-these-factors-matter}

The development potential score is calculated by weighing each of the
seven criteria according to its importance in determining development
viability. Our scoring methodology, as outlined in the analytical
framework of this report, reflects decades of real estate development
experience and incorporates both market realities and mission-aligned
priorities. Each criterion was selected to balance financial viability
with community impact, ensuring that recommended properties can support
sustainable development while serving congregational needs.

\begin{tcolorbox}[enhanced jigsaw, breakable, bottomrule=.15mm, title=\textcolor{quarto-callout-tip-color}{\faLightbulb}\hspace{0.5em}{Tip}, toprule=.15mm, opacitybacktitle=0.6, toptitle=1mm, left=2mm, bottomtitle=1mm, coltitle=black, titlerule=0mm, colbacktitle=quarto-callout-tip-color!10!white, colback=white, leftrule=.75mm, arc=.35mm, colframe=quarto-callout-tip-color-frame, rightrule=.15mm, opacityback=0]

This is just a starting point based on HDA's ideal of easy development
for congregations. If VEREP identifies other criteria they would like to
weigh differently, or has anecdotal information to add this scoring
could be adjusted.

\end{tcolorbox}

\subsection{Property Size: The Goldilocks
Principle}\label{property-size-the-goldilocks-principle}

\textbf{Weight: 20\%}

\marginnote{\begin{footnotesize}

\textbf{Optimal Range}: 0.5-5 acres

Properties that are ``just right'' for development.

\end{footnotesize}}

Not too small to be economically viable, not too large to be financially
unwieldy.

\begin{figure}[H]

\centering{

\pandocbounded{\includegraphics[keepaspectratio]{development-criteria_files/figure-pdf/fig-size-scoring-1.pdf}}

}

\caption{\label{fig-size-scoring}Property Size Scoring Curve}

\end{figure}%

\begin{figure}[H]

\centering{

\pandocbounded{\includegraphics[keepaspectratio]{development-criteria_files/figure-pdf/fig-size-distribution-1.pdf}}

}

\caption{\label{fig-size-distribution}Distribution of Property Sizes in
Portfolio}

\end{figure}%

\textbf{Why This Matters}:

\begin{itemize}
\tightlist
\item
  \textbf{Too small (\textless0.5 acres)}: Limited development options,
  high per-unit costs
\item
  \textbf{Optimal (0.5-5 acres)}: Maximum flexibility and economic
  efficiency
\item
  \textbf{Too large (\textgreater10 acres)}: Requires significant
  capital, complex phasing
\end{itemize}

Property size significantly influences development feasibility, but
bigger isn't always better. Properties under a quarter-acre typically
lack the critical mass needed for financially viable
development---construction costs per unit rise sharply, and parking or
open space requirements become impossible to meet. Conversely, parcels
exceeding 10 acres often present unique challenges: they may require
complex phasing, demand substantial upfront capital, or face community
resistance to density.

\subsection{Current Use: Identifying Low-Hanging
Fruit}\label{current-use-identifying-low-hanging-fruit}

\textbf{Weight: 25\%}

\marginnote{\begin{footnotesize}

\textbf{Optimal Use}: Vacant or nearly vacant with access to
infrastructure

\end{footnotesize}}

Properties with minimal current utilization represent immediate
opportunities.

\begin{longtable}[t]{l>{}cl}

\caption{\label{tbl-use-scoring}Current Use Scoring Matrix}

\tabularnewline

\toprule
Current Use & Score & Rationale\\
\midrule
Parking Lot & \cellcolor{RdYlGn}{\textcolor{black}{\textbf{95}}} & Immediate development potential, minimal displacement\\
Open Space & \cellcolor{RdYlGn}{\textcolor{black}{\textbf{90}}} & Low-value use, easy conversion\\
Mixed/Other & \cellcolor{RdYlGn}{\textcolor{black}{\textbf{60}}} & Varies by specific circumstances\\
Residence & \cellcolor{RdYlGn}{\textcolor{black}{\textbf{40}}} & Displacement considerations\\
School & \cellcolor{RdYlGn}{\textcolor{black}{\textbf{35}}} & Active community use\\
\addlinespace
Church (primary) & \cellcolor{RdYlGn}{\textcolor{black}{\textbf{30}}} & Core mission function\\
Cemetery & \cellcolor{RdYlGn}{\textcolor{black}{\textbf{5}}} & Restricted use, not developable\\
\bottomrule

\end{longtable}

\begin{figure}[H]

\centering{

\pandocbounded{\includegraphics[keepaspectratio]{development-criteria_files/figure-pdf/fig-use-distribution-1.pdf}}

}

\caption{\label{fig-use-distribution}Properties by Current Use Type}

\end{figure}%

\textbf{Highest Potential: Parking Lots and Open Space}

Not all church property serves its highest and best use. Surface parking
lots represent prime redevelopment opportunities---they're already
cleared and graded, typically have minimal environmental constraints,
and often sit underutilized six days per week. A parking lot that serves
200 congregants on Sunday morning but stands empty Monday through
Saturday represents an enormous opportunity cost.

Open space scores similarly high, particularly when it's amenity-free
lawn rather than programmed recreation or memorial gardens. These
properties can be re-imagined while potentially retaining some green
space within new development.

\subsection{Location Quality: Following the
Market}\label{location-quality-following-the-market}

\textbf{Weight: 20\%}

\begin{figure}[H]

\centering{

\pandocbounded{\includegraphics[keepaspectratio]{development-criteria_files/figure-pdf/fig-location-factors-1.pdf}}

}

\caption{\label{fig-location-factors}Location Quality Components}

\end{figure}%

\textbf{Key Metrics: Walkability Score and Transit Access}

Real estate development fundamentally responds to location. A walkable,
transit-rich site commands higher rents, attracts more diverse tenants,
and often qualifies for density bonuses or reduced parking
requirements---all factors that improve project economics.

Walkability scores measure proximity to everyday needs: grocery stores,
schools, employment centers, healthcare. High walkability (15-20 on our
scale) indicates a property can support car-light or car-free
households, expanding the potential tenant base to include young
professionals, seniors aging in place, and lower-income families for
whom car ownership is cost-prohibitive.

Transit access amplifies these benefits. Properties within a
quarter-mile of frequent bus service or half-mile of rail stations can
often negotiate reduced parking requirements with municipalities---a
significant cost savings when structured parking runs \$25,000-\$40,000
per space to construct.

Most parcels in the portfolio fall below the high-walkability range, but
a number showed promise in this category.

\begin{figure}[H]

\centering{

\pandocbounded{\includegraphics[keepaspectratio]{development-criteria_files/figure-pdf/fig-walkability-dist-1.pdf}}

}

\caption{\label{fig-walkability-dist}Distribution of Walkability Scores}

\end{figure}%

\subsection{Financial Need: Aligning Development with
Mission}\label{sec-FinancialNeed}

\textbf{Weight: 15\%}

\begin{figure}[H]

\centering{

\pandocbounded{\includegraphics[keepaspectratio]{development-criteria_files/figure-pdf/fig-financial-need-1.pdf}}

}

\caption{\label{fig-financial-need}Properties by Financial Need Level}

\end{figure}%

\textbf{Priority: Congregations with Declining Resources}

This criterion explicitly centers mission over pure market return.
Congregations experiencing sustained pledge declines often face a
difficult reality: their historic buildings demand expensive
maintenance, but shrinking budgets make upkeep increasingly burdensome.
Many such congregations sit on valuable real estate that could, through
thoughtful development, generate steady income streams while maintaining
their worship and ministry.

A ground lease arrangement, for example, might provide a struggling
congregation with \$50,000-\$150,000 annually in stable income---enough
to fund a part-time rector, maintain the building, and sustain core
ministries. This approach transforms real estate from a drain on
resources into a mission enabler.

Properties associated with growing congregations score lower not because
they lack development potential, but because the urgency is less acute.
These communities likely have more options and less immediate financial
pressure.

\subsection{Market Potential: Reading Economic
Signals}\label{market-potential-reading-economic-signals}

\textbf{Weight: 10\%}

\begin{table}

\caption{\label{tbl-market-indicators}Market Potential Scoring Criteria}

\centering{

\begin{longtabu} to \linewidth {>{\raggedright}X>{\centering}X>{\raggedright}X>{\raggedright}X}
\toprule
Indicator & Weight & High Score & Data Source\\
\midrule
QCT Status & 40\% & In Qualified Census Tract & HUD QCT Maps\\
Area Median Income & 30\% & >\$75,000 AMI & ACS 5-Year\\
Population Growth & 15\% & >2\% annual growth & Census\\
Housing Demand & 15\% & High demand market & Local market data\\
\bottomrule
\end{longtabu}

}

\end{table}%

\textbf{Indicators: Area Median Income, LIHTC Eligibility, and Household
Income Growth Forecast}

Development must respond to market demand. Properties in higher-income
areas (median incomes above \$75,000) typically support market-rate
housing, ground-floor retail, or mixed-use projects that can
cross-subsidize affordable components. These projects attract
conventional financing and a broader range of development partners.

Properties in Qualified Census Tracts (QCTs) gain additional scoring
weight because they unlock Low-Income Housing Tax Credit (LIHTC)
financing---the nation's primary mechanism for affordable housing
production. LIHTC projects in QCTs receive point advantages in
competitive funding rounds, improving their feasibility. For
mission-driven institutions committed to affordable housing, QCT
properties offer a rare alignment of social impact and financial
viability.

Median income tells you where a market is today---but real estate
development takes 3-5 years from concept to occupancy. Income growth
forecast tells you where the market is heading, which is critical for
timing development decisions

\subsection{Zoning: Navigating Regulatory
Reality}\label{zoning-navigating-regulatory-reality}

\textbf{Weight: 10\%}

\begin{table}

\caption{\label{tbl-zoning-scoring}Zoning Category Scoring}

\centering{

\begin{longtabu} to \linewidth {>{\raggedright}X>{}c>{\raggedright}X}
\toprule
Zoning Category & Score & Development Impact\\
\midrule
Mixed Use & \textbf{90} & Maximum flexibility for housing + retail\\
Multi-Family Residential & \textbf{85} & Aligned with housing mission\\
Commercial & \textbf{70} & May require use variance for residential\\
Single-Family Residential & \textbf{50} & Density limitations, rezoning often needed\\
Agricultural & \textbf{40} & Significant restrictions, rural areas\\
\addlinespace
Conservation & \textbf{10} & Severe limitations, unlikely candidate\\
\bottomrule
\end{longtabu}

}

\end{table}%

\textbf{Favorable: Mixed-Use, Commercial, and Residential Districts}

Zoning powerfully shapes what's buildable and how quickly projects move
from concept to construction. Properties zoned for residential or
mixed-use development typically offer by-right development
opportunities---projects that don't require lengthy rezoning processes,
conditional use permits, or special exceptions.

Conversely, properties in restrictive single-family or conservation
districts may require multi-year regulatory processes, neighborhood
opposition, and uncertain outcomes. While such projects occasionally
succeed, they demand patient capital and sophisticated development
partners.

\subsection{Environmental and Regulatory Constraints: Deal-Breakers
vs.~Speed-Bumps}\label{environmental-and-regulatory-constraints-deal-breakers-vs.-speed-bumps}

\textbf{Weight: 5\% (applied as penalties)}

\begin{table}

\caption{\label{tbl-constraints}Constraint Penalties Applied to
Development Score}

\centering{

\begin{longtabu} to \linewidth {>{\raggedright}X>{}c>{\centering}X>{\raggedright}X}
\toprule
Constraint & Score Penalty & Mitigation Difficulty & Notes\\
\midrule
100-Year Flood Zone & \textcolor[HTML]{e76f52}{\textbf{-40}} & High & Expensive insurance, elevated construction\\
Significant Wetlands (>25\%) & \textcolor[HTML]{e76f52}{\textbf{-35}} & High & Federal/state permitting, mitigation required\\
Easements Present & \textcolor[HTML]{e76f52}{\textbf{-30}} & Moderate & May restrict building envelope\\
Historic District & \textcolor[HTML]{e76f52}{\textbf{-25}} & Moderate & Design review, material requirements\\
Moderate Wetlands (10-25\%) & \textcolor[HTML]{e76f52}{\textbf{-20}} & Low-Moderate & Potential buildable area reduction\\
\bottomrule
\end{longtabu}

}

\end{table}%

\begin{tcolorbox}[enhanced jigsaw, breakable, bottomrule=.15mm, title=\textcolor{quarto-callout-note-color}{\faInfo}\hspace{0.5em}{Assessment Framework}, toprule=.15mm, opacitybacktitle=0.6, toptitle=1mm, left=2mm, bottomtitle=1mm, coltitle=black, titlerule=0mm, colbacktitle=quarto-callout-note-color!10!white, colback=white, leftrule=.75mm, arc=.35mm, colframe=quarto-callout-note-color-frame, rightrule=.15mm, opacityback=0]

\textbf{Deal-Breakers} (Score: 0-30):

\begin{itemize}
\tightlist
\item
  Extensive wetlands covering majority of site
\item
  Active floodway designation
\item
  Conservation easements prohibiting development
\item
  Contaminated sites requiring remediation
\end{itemize}

\textbf{Speed-Bumps} (Score: 60-90):

\begin{itemize}
\tightlist
\item
  Partial flood zone (can design around)
\item
  Historic considerations (design flexibility)
\item
  Minor wetlands (avoidable)
\item
  Standard permitting requirements
\end{itemize}

\end{tcolorbox}

These constraints receive substantial score penalties because they
materially affect project feasibility and cost. Properties in 100-year
floodplains face expensive flood insurance, elevated construction costs,
and increasingly cautious lenders post-climate-change. Development in
such areas may be technically possible but financially marginal.

Significant wetlands (\textgreater25\% of parcel) trigger federal and
state permitting, potential mitigation requirements, and uncertainty
about buildable area. Easements---particularly those held by third
parties---can restrict development rights, limit building envelopes, or
prevent property subdivision.

Historic district designation doesn't prohibit development but adds
layers of design review, material requirements, and timeline
uncertainty. Some historic commissions embrace creative contemporary
additions; others mandate strict historicism that may conflict with
modern construction economics.

\section{Bringing It All Together}\label{bringing-it-all-together}

The final Development Potential Score is calculated as:

\[
\text{Score} = \sum_{i=1}^{6} (w_i \times s_i) - \text{Constraint Penalties}
\]

Where:

\begin{itemize}
\tightlist
\item
  \(w_i\) = weight of criterion \(i\)
\item
  \(s_i\) = score for criterion \(i\) (0-100)
\end{itemize}

Simple example:

\begin{longtable*}[t]{lccc}
\toprule
\cellcolor[HTML]{445ca9}{\textcolor{white}{\textbf{Criterion}}} & \cellcolor[HTML]{445ca9}{\textcolor{white}{\textbf{Weight}}} & \cellcolor[HTML]{445ca9}{\textcolor{white}{\textbf{Score}}} & \cellcolor[HTML]{445ca9}{\textcolor{white}{\textbf{Weighted Score}}}\\
\midrule
Land value & 0.3 & 80 & 24\\
Acreage & 0.25 & 60 & 15\\
Location & 0.2 & 90 & 18\\
Zoning & 0.15 & 70 & 10.5\\
Accessibility & 0.1 & 50 & 5\\
\addlinespace
\cellcolor[HTML]{f0f0f0}{\textbf{**Total**}} & \cellcolor[HTML]{f0f0f0}{\textbf{—}} & \cellcolor[HTML]{f0f0f0}{\textbf{—}} & \cellcolor[HTML]{f0f0f0}{\textbf{**72.5**}}\\
\bottomrule
\end{longtable*}

If there's a constraint penalty of 10 points on this parcel (say, for a
historic easement), the final score would be \textbf{62.5}

\begin{figure}[H]

\centering{

\pandocbounded{\includegraphics[keepaspectratio]{development-criteria_files/figure-pdf/fig-score-components-1.pdf}}

}

\caption{\label{fig-score-components}Development Score Component
Weights}

\end{figure}%

\textbf{Score Interpretation}:

\begin{longtable}[]{@{}
  >{\raggedright\arraybackslash}p{(\linewidth - 4\tabcolsep) * \real{0.4062}}
  >{\raggedright\arraybackslash}p{(\linewidth - 4\tabcolsep) * \real{0.1875}}
  >{\raggedright\arraybackslash}p{(\linewidth - 4\tabcolsep) * \real{0.4062}}@{}}
\toprule\noalign{}
\begin{minipage}[b]{\linewidth}\raggedright
Score Range
\end{minipage} & \begin{minipage}[b]{\linewidth}\raggedright
Tier
\end{minipage} & \begin{minipage}[b]{\linewidth}\raggedright
Description
\end{minipage} \\
\midrule\noalign{}
\endhead
\bottomrule\noalign{}
\endlastfoot
75-100 & Tier 1 & \textbf{High Priority} - Immediate development
candidates \\
60-74 & Tier 2 & \textbf{Strong Potential} - Near-term opportunities \\
45-59 & Tier 3 & \textbf{Moderate Potential} - Requires creative
solutions \\
30-44 & Tier 4 & \textbf{Limited Potential} - Significant barriers
exist \\
\textless30 & Tier 5 & \textbf{Not Recommended} - Unsuitable for
development \\
\end{longtable}

These seven criteria create a holistic picture of development potential.
A property might score exceptionally well on size and location but face
significant challenges from environmental constraints. Another might
have modest physical attributes but represent an urgent opportunity to
support a financially stressed congregation.

The weighted scoring system allows us to compare apples to oranges---to
evaluate whether a small, perfectly-located property in a historic
district outweighs a larger, unrestricted site with less market demand.
This methodology doesn't make decisions for stakeholders but rather
creates a common language for discussing trade-offs and priorities.

Ultimately, successful church real estate development requires more than
high scores. It demands patient capital, mission-aligned partners,
engaged congregations, and creative design. But by systematically
evaluating these criteria, we can identify the properties where stars
may align---where market demand, congregational need, and regulatory
environment likely converge to create genuine opportunity.

\part{Analysis Results}

\chapter{Development Opportunity
Landscape}\label{development-opportunity-landscape}

Of the 104 parcels pulled from the VEREP GIS tool for examination, 4
were immediately removed from development opportunity consideration due
to their size. These parcels were all under 0.1 acres -- likely cemetery
plots, parking strips, or slivers of land. Three of these tiny parcels
belong to the same congregation, so while not candidates for independent
development, they may hold value as part of a larger assemblage.

\begin{longtable*}[t]{lllrr}
\toprule
Property & Congregation & County & Acres & Sq Feet\\
\midrule
St Johns Episcopal Church & St Johns Episcopal Church & KING-WILLIAM & 0.066 & 2,871\\
St Johns Episcopal Church & St Johns Episcopal Church & KING-WILLIAM & 0.068 & 2,955\\
St Pauls Church & St Pauls Church & KING-WILLIAM & 0.027 & 1,185\\
St Johns Episcopal Church & St Johns Episcopal Church & KING-WILLIAM & 0.066 & 2,878\\
\bottomrule
\end{longtable*}

\textbf{This means ultimately only 100 parcels were scored and ranked
for development opportuntity.}

\pandocbounded{\includegraphics[keepaspectratio]{development-landscape_files/figure-pdf/tier-distribution-visual-1.pdf}}

Of the 100 properties analyzed with congregations under 30 Sunday
attendees, \textbf{39 properties (39\%) fall within the ideal
development range} of chosen criteria.

\begin{itemize}
\tightlist
\item
  \textbf{High Priority}: 16 properties (16\%) --- 1.5-5 acres with
  optimal development characteristics
\item
  \textbf{Moderate Priority}: 23 properties (23\%) --- 0.5-1.5 acres
  with development potential
\end{itemize}

The remaining 61\% face size or constraint limitations: \textbf{35
properties (35\%) are small parcels} under 0.5 acres offering limited
standalone development options; \textbf{11 properties (11\%) are too
large} (\textgreater5 acres) requiring phased development or specialized
partners; and \textbf{15 properties (15\%) are environmentally
constrained} by wetlands, flood zones, or high natural hazard risk.

This distribution reveals that while a significant minority of the
portfolio presents actionable development opportunities, most properties
face fundamental size constraints---either too small for economical
development or too large for congregation-scale management.

\chapter{Geographic Distribution}\label{geographic-distribution-1}

\section{Interactive Map}\label{interactive-map}

\textbf{Development opportunities are distributed broadly across the
Diocese, with the color gradient revealing where the highest-scoring
properties cluster.}

The Diocese's strongest opportunities, scoring in the upper 70s, are
concentrated in \textbf{Winchester in the northern Shenandoah Valley},
where two properties at St.~Paul's on the Hill Church achieved the
portfolio's highest development scores. These sites benefit from optimal
acreage (1.5-3 acres), favorable market conditions, and minimal
environmental or regulatory constraints.

Moderate-to-strong opportunities (scores 50-65), shown in orange to
light red, are scattered throughout \textbf{Northern Virginia/DC
corridor near Alexandria and Fairfax}, as well as in the \textbf{greater
Richmond area}. These properties fall within the goldilocks acreage
range but face scoring penalties from historic district designations
(affecting 45\% of portfolio properties) or moderate natural hazard
risk. While viable for development, they will require additional time
for regulatory approvals and design review processes.

The \textbf{Fredericksburg and Central Virginia regions} show numerous
properties scoring in the 40-55 range, represented by medium-sized
circles in orange tones. Many of these sites have favorable acreage but
receive constraint penalties that reduce their immediate development
appeal. However, they remain viable candidates for phased development or
projects with patient capital willing to navigate historic preservation
requirements.

\begin{tcolorbox}[enhanced jigsaw, breakable, bottomrule=.15mm, title=\textcolor{quarto-callout-tip-color}{\faLightbulb}\hspace{0.5em}{Tip}, toprule=.15mm, opacitybacktitle=0.6, toptitle=1mm, left=2mm, bottomtitle=1mm, coltitle=black, titlerule=0mm, colbacktitle=quarto-callout-tip-color!10!white, colback=white, leftrule=.75mm, arc=.35mm, colframe=quarto-callout-tip-color-frame, rightrule=.15mm, opacityback=0]

This geographic diversity is strategic: the Diocese can pursue the
highest-scoring ``flagship'' opportunities in Winchester for near-term
demonstration projects, while simultaneously advancing portfolio-scale
development in Northern Virginia where multiple properties could support
coordinated initiatives. Larger or more constrained properties elsewhere
provide longer-term strategic options as markets evolve and regulatory
environments change.

\end{tcolorbox}

\emph{The interactive map is available in the HTML version of this
report.}

\part{Summary Statistics}

\chapter{Property Portfolio Overview}\label{property-portfolio-overview}

\section{Summary Metrics}\label{summary-metrics}

\begin{figure}

\begin{minipage}{0.25\linewidth}

\begin{tcolorbox}[enhanced jigsaw, breakable, left=2mm, bottomrule=.15mm, opacityback=0, colback=white, leftrule=.75mm, arc=.35mm, colframe=quarto-callout-note-color-frame, rightrule=.15mm, toprule=.15mm]

\vspace{-3mm}\textbf{104}\vspace{3mm}

Total Parcels Analyzed

\end{tcolorbox}

\end{minipage}%
%
\begin{minipage}{0.25\linewidth}

\begin{tcolorbox}[enhanced jigsaw, breakable, left=2mm, bottomrule=.15mm, opacityback=0, colback=white, leftrule=.75mm, arc=.35mm, colframe=quarto-callout-tip-color-frame, rightrule=.15mm, toprule=.15mm]

\vspace{-3mm}\textbf{248}\vspace{3mm}

Total Acres Analyzed

\end{tcolorbox}

\end{minipage}%
%
\begin{minipage}{0.25\linewidth}

\begin{tcolorbox}[enhanced jigsaw, breakable, left=2mm, bottomrule=.15mm, opacityback=0, colback=white, leftrule=.75mm, arc=.35mm, colframe=quarto-callout-warning-color-frame, rightrule=.15mm, toprule=.15mm]

\vspace{-3mm}\textbf{35.3}\vspace{3mm}

Avg Development Score

\end{tcolorbox}

\end{minipage}%
%
\begin{minipage}{0.25\linewidth}

\begin{tcolorbox}[enhanced jigsaw, breakable, left=2mm, bottomrule=.15mm, opacityback=0, colback=white, leftrule=.75mm, arc=.35mm, colframe=quarto-callout-important-color-frame, rightrule=.15mm, toprule=.15mm]

\vspace{-3mm}\textbf{16}\vspace{3mm}

High Potential Properties

\end{tcolorbox}

\end{minipage}%

\end{figure}%

The distribution of development potential reflected in this analysis is
common in institutional portfolios where properties were acquired for
mission, not market value. The analysis's value lies in identifying the
needles in the haystack---the 16 high-priority and 23 moderate-priority
properties that can actually deliver financial returns---rather than
pretending the entire portfolio has equal potential.

\textbf{In short:} A median development score of 35 proves the
methodology is honest, makes the high-scoring properties more valuable
by contrast, and provides clear strategic guidance: focus resources
where they'll succeed rather than spreading them thin across a portfolio
where most properties face fundamental limitations.

\section{Property Use Types}\label{property-use-types}

\begin{figure}[H]

\centering{

\pandocbounded{\includegraphics[keepaspectratio]{property-portfolio_files/figure-pdf/fig-use-types-1.pdf}}

}

\caption{\label{fig-use-types}Distribution of Properties by Primary Use}

\end{figure}%

Over two-thirds (71\%) of parcels are locked into church and cemetery
designations---uses that inherently limit redevelopment options due to
zoning, community sensitivity, and legal restrictions. Only about 29\%
of the portfolio consists of property types (open space, parking,
residence) that would typically command higher valuations or offer more
flexible development opportunities.

\section{Development Score
Distribution}\label{development-score-distribution}

The development scoring confirms use-constraints. Focus should likely
center on the 16 high-potential properties for near-term value creation,
while developing a longer-term strategy for the 23 moderate-potential
parcels. The 59 constrained and size-restrictive properties may require
hold strategies, community partnerships, or creative repurposing rather
than traditional development approaches.

\begin{figure}[H]

\centering{

\pandocbounded{\includegraphics[keepaspectratio]{property-portfolio_files/figure-pdf/fig-score-distribution-1.pdf}}

}

\caption{\label{fig-score-distribution}Distribution of Development
Scores Across Portfolio}

\end{figure}%

\section{Property Size vs Development
Score}\label{property-size-vs-development-score}

\begin{figure}[H]

\centering{

\pandocbounded{\includegraphics[keepaspectratio]{property-portfolio_files/figure-pdf/fig-size-vs-score-1.pdf}}

}

\caption{\label{fig-size-vs-score}Development Scores by Property Size
Category}

\end{figure}%

The most high and moderate-potential properties concentrate in the
0.5--5 acre ``sweet spot,'' while parcels under 0.5 acre face inherent
limitations and the largest sites show more constraints.

\section{Summary Narrative}\label{summary-narrative}

\subsection{By County}\label{by-county}

\subsubsection{Geographic Distribution: Understanding Two Complementary
Metrics}\label{geographic-distribution-understanding-two-complementary-metrics}

The parcels of interest in this portfolio span multiple counties with
highly variable concentrations and development potentials. Thanks to our
analysis method, we can measure different aspects of development
potential:

\textbf{``High Potential'' Classification} answers: \emph{``Can this be
developed?''} It's a simple yes/no based on physical requirements---the
property must be 1.5--5 acres, have few environmental constraints, and
have documented land value. Think of it as a checklist: pass or fail.

\textbf{Development Score (0--100)} answers: \emph{``How attractive is
this for development?''} It's a weighted score considering multiple
factors. A property might be physically developable but score low
because the church is financially healthy (less need) or the market is
weak.

These metrics can diverge significantly, revealing different types of
opportunities:

\begin{longtable}[]{@{}
  >{\raggedright\arraybackslash}p{(\linewidth - 6\tabcolsep) * \real{0.2381}}
  >{\raggedright\arraybackslash}p{(\linewidth - 6\tabcolsep) * \real{0.3810}}
  >{\raggedright\arraybackslash}p{(\linewidth - 6\tabcolsep) * \real{0.1667}}
  >{\raggedright\arraybackslash}p{(\linewidth - 6\tabcolsep) * \real{0.2143}}@{}}
\toprule\noalign{}
\begin{minipage}[b]{\linewidth}\raggedright
Scenario
\end{minipage} & \begin{minipage}[b]{\linewidth}\raggedright
High Potential?
\end{minipage} & \begin{minipage}[b]{\linewidth}\raggedright
Score
\end{minipage} & \begin{minipage}[b]{\linewidth}\raggedright
Meaning
\end{minipage} \\
\midrule\noalign{}
\endhead
\bottomrule\noalign{}
\endlastfoot
Great site, struggling church, strong market & ✓ Yes & 85 & Top
priority \\
Great site, thriving church, weak market & ✓ Yes & 45 & Developable but
less urgent/attractive \\
Too small, struggling church & ✗ No & 60 & High need but site doesn't
qualify \\
\end{longtable}

Looking at both metrics together helps identify properties that are
\textbf{both} physically developable \textbf{and} strategically worth
pursuing.

\textbf{Tier 1: Premium Opportunities (High Scores + High Potential)}

\begin{itemize}
\item
  Frederick County (Winchester): The portfolio's standout performer
\item
  Albemarle \& Fairfax Counties: Strong individual opportunities

  These counties offer proven development candidates with favorable
  market conditions offsetting some regulatory complexity (historic
  districts in several cases)
\end{itemize}

\textbf{Tier 2: Volume Opportunities (High Potential Count, Modest
Scores)}

\begin{itemize}
\item
  Fauquier County: The geographic concentration story

  These properties have the right physical characteristics (good
  acreage, unconstrained) but face market and location limitations
  (rural setting, lower walkability scores, modest income
  growth).Multiple properties here could support coordinated affordable
  housing or creative mixed-use projects with patient capital.
\item
  Fairfax-City: Urban properties with modest outcomes

  The moderate scores despite good market fundamentals suggest
  regulatory complexity (likely historic districts) or current use
  challenges (active church buildings rather than parking lots) These
  would require development partners comfortable with urban regulatory
  environments.
\end{itemize}

\textbf{Tier 3: The King William Challenge (High Volume, Low Potential)}

\begin{itemize}
\item
  King William County: The inverse of Fauquier AKA geographic spread

  King William contains 18\% of all the parcels in this analysis but
  contributes zero strong development opportunities. These properties
  fail the physical threshold and likely represent legacy
  holdings---land accumulated through historical donations and gifts
  that don't always align with modern development economics. Consider
  disposition, conservation easements, or alternative mission uses
  (community gardens, land banks) rather than pursuing development here.
\end{itemize}

\textbf{Tier 4: Limited Engagement Counties}

\begin{itemize}
\tightlist
\item
  Richmond, Fluvanna, and others: Single properties with mixed potential
\end{itemize}

These counties offer opportunistic development where individual
properties merit attention but don't justify significant resource
concentration. We suggest VEREP evaluate these individually as
opportunities arise.

\subsubsection{Key Strategic Insights:}\label{key-strategic-insights}

\textbf{1. Multiple parcels ≠ multiple opportunities}

Parcel concentration isn't inherently advantageous. A county with many
parcels may simply face a larger legacy management challenge---more
aging buildings, more deferred maintenance, more administrative
complexity---rather than greater development potential.

What matters is the concentration of \emph{opportunity}. A cluster of
``High Potential'' properties in a single geography enables economies of
scale: coordinated development planning, shared infrastructure
investments, streamlined partner recruitment, and consolidated
regulatory engagement. Scattered high-potential sites or concentrated
low-potential sites don't offer these efficiencies.

Many parcels serve single congregations. The 104 total parcels overstate
discrete development sites---the true count is closer to 39 high and
moderate potential opportunities across fewer than 20 congregational
sites.

\textbf{2. Resource allocation follows quality, not quantity}

Staff time, consulting budgets, feasibility studies, and partner
outreach should flow toward sites with real potential---not spread thin
across a large portfolio simply because it exists. Managing many
low-potential properties is necessary work, but it's maintenance, not
strategy. Concentrating resources where development can actually happen
maximizes impact and avoids the trap of mistaking activity for progress.

\textbf{Bottom Line:} The Diocese should pursue an uneven geographic
strategy that concentrates development resources in Frederick/Winchester
(flagship quality), Fauquier (portfolio scale), and select
Albemarle/Fairfax opportunities (proven viability), while being
realistic that high-volume counties like King William represent asset
management challenges, not development opportunities.

\subsection{High Potential Parcels}\label{high-potential-parcels}

\begin{longtable}[t]{clllrrl}

\caption{\label{tbl-high-potential}Top Development Potential Parcels}

\tabularnewline

\toprule
\cellcolor[HTML]{445ca9}{\textcolor{white}{\textbf{Rank}}} & \cellcolor[HTML]{445ca9}{\textcolor{white}{\textbf{Address}}} & \cellcolor[HTML]{445ca9}{\textcolor{white}{\textbf{City}}} & \cellcolor[HTML]{445ca9}{\textcolor{white}{\textbf{County}}} & \cellcolor[HTML]{445ca9}{\textcolor{white}{\textbf{Acres}}} & \cellcolor[HTML]{445ca9}{\textcolor{white}{\textbf{Score}}} & \cellcolor[HTML]{445ca9}{\textcolor{white}{\textbf{Tier}}}\\
\midrule
\textbf{1} & \textbf{1527 SENSENY RD, WINCHESTER 22602} & \textbf{WINCHESTER} & \textbf{FREDERICK} & \textbf{3.20} & \textbf{79.2} & \textbf{High}\\
\textbf{2} & \textbf{WINCHESTER 22602} & \textbf{WINCHESTER} & \textbf{FREDERICK} & \textbf{1.67} & \textbf{76.0} & \textbf{High}\\
\textbf{3} & \textbf{7420 MISSION HOME RD, FREE UNION 22940} & \textbf{FREE UNION} & \textbf{ALBEMARLE} & \textbf{3.12} & \textbf{63.2} & \textbf{High}\\
\textbf{4} & \textbf{5096 GRACE CHURCH LN, MIDLAND 22728} & \textbf{MIDLAND} & \textbf{FAUQUIER} & \textbf{3.65} & \textbf{58.2} & \textbf{High}\\
\textbf{5} & \textbf{5094 GRACE CHURCH LN, MIDLAND 22728} & \textbf{MIDLAND} & \textbf{FAUQUIER} & \textbf{2.76} & \textbf{58.2} & \textbf{High}\\
\addlinespace
6 & 3392 PINE GROVE RD, STANLEY 22851 & STANLEY & PAGE & 1.83 & 57.0 & High\\
7 & 1 TRURO LN, FAIRFAX 22030 & FAIRFAX & FAIRFAX-CITY & 3.44 & 52.8 & High\\
8 & STAFFORD 22554 & STAFFORD & STAFFORD & 1.53 & 50.8 & High\\
9 & 3421 FRANCONIA RD, ALEXANDRIA 22310 & ALEXANDRIA & FAIRFAX & 4.39 & 50.2 & High\\
10 & 3421 FRANCONIA RD, ALEXANDRIA 22310 & ALEXANDRIA & FAIRFAX & 1.67 & 48.2 & High\\
\bottomrule

\end{longtable}

\section{Congregation Health
Indicators}\label{congregation-health-indicators}

\begin{figure}[H]

\centering{

\pandocbounded{\includegraphics[keepaspectratio]{property-portfolio_files/figure-pdf/fig-attendance-pledge-1.pdf}}

}

\caption{\label{fig-attendance-pledge}Congregation Health and Financial
Need}

\end{figure}%

\begin{tcolorbox}[enhanced jigsaw, breakable, bottomrule=.15mm, title=\textcolor{quarto-callout-note-color}{\faInfo}\hspace{0.5em}{Understanding The Financial Need Spread}, toprule=.15mm, opacitybacktitle=0.6, toptitle=1mm, left=2mm, bottomtitle=1mm, coltitle=black, titlerule=0mm, colbacktitle=quarto-callout-note-color!10!white, colback=white, leftrule=.75mm, arc=.35mm, colframe=quarto-callout-note-color-frame, rightrule=.15mm, opacityback=0]

This chart reveals a counterintuitive finding: congregations flagged as
``Financial Need'' often look similar to ``Stable'' congregations on
paper. Many orange triangles (Financial Need) sit alongside blue circles
(Stable) in the upper-right quadrant---healthy attendance, strong
giving.

The difference isn't current performance; it's trajectory. ``Financial
Need'' congregations have experienced significant declines in attendance
or pledge revenue since 2014, even if their present numbers still appear
respectable. They're not in crisis today, but the trend line points
toward one.

This distinction matters strategically. Development partnerships are
easier to negotiate---and more likely to succeed---while a congregation
still has resources, leadership capacity, and time to plan. Waiting
until finances are exhausted limits options and leverage. The goal is
intervention during slow decline, not rescue after collapse.

\end{tcolorbox}

\begin{longtable*}[t]{lrrrr}
\toprule
\cellcolor[HTML]{445ca9}{\textcolor{white}{\textbf{Congregation}}} & \cellcolor[HTML]{445ca9}{\textcolor{white}{\textbf{Avg Attendance}}} & \cellcolor[HTML]{445ca9}{\textcolor{white}{\textbf{Avg Pledge}}} & \cellcolor[HTML]{445ca9}{\textcolor{white}{\textbf{Attendance Change}}} & \cellcolor[HTML]{445ca9}{\textcolor{white}{\textbf{Pledge Change}}}\\
\midrule
Church of Our Redeemer & 19 & \$53,675 & -78.9\% & -68.3\%\\
Emmanuel Church & 25 & \$74,169 & -57.6\% & -43.2\%\\
St Johns Episcopal Church & 21 & \$96,947 & -55.3\% & -26.8\%\\
Grace Church Emmanuel Parish & 14 & \$72,564 & -51.7\% & -18.1\%\\
Piedmont/Bromfield Parish & 29 & \$70,810 & -51.7\% & -25.8\%\\
\addlinespace
St Stephens Church & 23 & \$101,136 & -50\% & -28.9\%\\
St James Church & 25 & \$55,882 & -21.9\% & -4.6\%\\
\bottomrule
\end{longtable*}

These congregations show the largest declines in attendance and pledge
over recent years.

Grace Church Emmanuel Parish in Fauquier County exemplifies the
strategic opportunity. It ranks among the top 10 properties by
development score, reports over \$50,000 in annual giving, yet shows a
declining trajectory that makes proactive planning urgent. Its location
in Fauquier---where multiple high-potential properties cluster---creates
the possibility of coordinated development efforts across sites.

\chapter{Geographic Summary}\label{geographic-summary}

\subsection{Parcels by Diocesan
Region}\label{parcels-by-diocesan-region}

\begin{longtable}[t]{lrrrr}

\caption{\label{tbl-by-county}Property Distribution by County}

\tabularnewline

\toprule
\cellcolor[HTML]{445ca9}{\textcolor{white}{\textbf{County}}} & \cellcolor[HTML]{445ca9}{\textcolor{white}{\textbf{Parcels}}} & \cellcolor[HTML]{445ca9}{\textcolor{white}{\textbf{Total Acres}}} & \cellcolor[HTML]{445ca9}{\textcolor{white}{\textbf{Avg Score}}} & \cellcolor[HTML]{445ca9}{\textcolor{white}{\textbf{\# High Potential}}}\\
\midrule
FAUQUIER & 7 & 21.8 & 44.4 & 4\\
ALBEMARLE & 3 & 12.6 & 54.3 & 2\\
FAIRFAX & 3 & 11.7 & 55.2 & 2\\
FAIRFAX-CITY & 3 & 8.2 & 46.2 & 2\\
FREDERICK & 2 & 4.9 & 77.6 & 2\\
\addlinespace
STAFFORD & 8 & 14.9 & 42.7 & 1\\
RICHMOND & 7 & 7.1 & 23.2 & 1\\
LOUDOUN & 2 & 3.2 & 38.2 & 1\\
PAGE & 2 & 2.3 & 44.2 & 1\\
KING-WILLIAM & 19 & 16.3 & 30.5 & 0\\
\addlinespace
FLUVANNA & 7 & 8.4 & 42.9 & 0\\
WESTMORELAND & 7 & 6.5 & 21.0 & 0\\
CAROLINE & 5 & 2.1 & 21.1 & 0\\
CLARKE & 5 & 2.5 & 25.0 & 0\\
KING-GEORGE & 5 & 77.4 & 29.3 & 0\\
\addlinespace
CULPEPER & 3 & 17.9 & 39.7 & 0\\
ROCKINGHAM & 3 & 9.5 & 41.3 & 0\\
SHENANDOAH & 3 & 1.7 & 22.4 & 0\\
ESSEX & 2 & 4.6 & 4.4 & 0\\
MADISON & 2 & 0.9 & 30.6 & 0\\
\addlinespace
RICHMOND-CITY & 2 & 1.9 & 55.8 & 0\\
ALEXANDRIA & 1 & 0.3 & 26.5 & 0\\
ARLINGTON & 1 & 0.3 & 50.2 & 0\\
GOOCHLAND & 1 & 9.5 & 45.5 & 0\\
HANOVER & 1 & 1.5 & 54.3 & 0\\
\addlinespace
\cellcolor[HTML]{f0f0f0}{\textbf{**TOTAL**}} & \cellcolor[HTML]{f0f0f0}{\textbf{104}} & \cellcolor[HTML]{f0f0f0}{\textbf{247.8}} & \cellcolor[HTML]{f0f0f0}{\textbf{35.3}} & \cellcolor[HTML]{f0f0f0}{\textbf{16}}\\
\bottomrule

\end{longtable}

\chapter{Congregation Vitality
Metrics}\label{congregation-vitality-metrics}

\begin{verbatim}
✓ Scoring complete

=== TIER DISTRIBUTION ===
# A tibble: 5 x 2
  development_tier               n
  <chr>                      <int>
1 Tier 4: Limited Potential     37
2 Tier 5: Not Recommended       37
3 Tier 3: Moderate Potential    24
4 Tier 2: Strong Potential       4
5 Tier 1: High Priority          2

=== GOLDILOCKS OPPORTUNITIES ===
Total High + Moderate: 39 (37.5%)

=== TOP 15 PROPERTIES BY SCORE ===
# A tibble: 15 x 7
    rank congregation_name scity area_acres development_score constraint_summary
   <int> <chr>             <chr>      <dbl>             <dbl> <chr>             
 1     1 St Pauls on the ~ WINC~      3.20               79.2 None              
 2     2 St Pauls on the ~ WINC~      1.67               76   None              
 3     3 Christ Ascension~ RICH~      1.47               64.8 None              
 4     4 Good Shepherd-of~ FREE~      3.12               63.2 None              
 5     5 St Johns Church   COLU~      0.516              62   None              
 6     6 Grace Church Emm~ MIDL~      3.65               58.2 None              
 7     7 Grace Church Emm~ MIDL~      2.76               58.2 None              
 8     8 St Georges Church STAN~      1.83               57   None              
 9     9 Aquia Church      STAF~      1.12               54.6 None              
10    10 Calvary Episcopa~ HANO~      1.45               54.3 None              
11    11 Holy Cross Korea~ FAIR~      3.44               52.8 Historic District 
12    12 Aquia Church      STAF~      1.53               50.8 Historic District 
13    13 Aquia Church      STAF~      0.974              50.6 None              
14    14 Aquia Church      STAF~      0.948              50.6 None              
15    15 All Saints Sharo~ ALEX~      4.39               50.2 None              
# i 1 more variable: development_potential <chr>

=== CONSTRAINT IMPACT ===
# A tibble: 1 x 5
  total with_constraints avg_score_unconstrained avg_score_constrained
  <int>            <int>                   <dbl>                 <dbl>
1    39               22                    56.4                  34.9
# i 1 more variable: score_difference <dbl>

=== CONSTRAINT TYPES IN GOLDILOCKS ===
# A tibble: 4 x 2
  constraint_summary             n
  <chr>                      <int>
1 None                          17
2 Historic District             15
3 Moderate Hazard Risk           6
4 Historic + Moderate Hazard     1

✓ Files saved:
  - property_profile_scored.rds
  - property_profile_scored.csv
  - verep_top_opportunities.csv
\end{verbatim}

\textbf{Key Congregation Metrics:}

{37} Congregations with \textless30 Attendance

{17} Avg Sunday Attendance

{\$44,271} Avg Annual Pledge

{21} Declining Pledges

{19} In Financial Need

{2.8} Avg Parcels per Congregation

\chapter{Developable Land Summary}\label{developable-land-summary}

\subsection{Comprehensive Developability
Analysis}\label{comprehensive-developability-analysis}

\begin{longtabu} to \linewidth {>{\raggedright}X>{\raggedleft}X>{\raggedleft}X>{\raggedleft}X>{\raggedleft}X>{\raggedleft}X>{\raggedleft}X}
\caption{\label{tab:developable-summary}Developable Properties by Congregation (Score ≥57)}\\
\toprule
Congregation & Parcels & Total Acres & Assessed Value & Avg Attendance & Avg Members & Mean Dev Score\\
\midrule
\cellcolor[HTML]{445ca9}{\textcolor{white}{St Pauls on the Hill Church}} & \cellcolor[HTML]{445ca9}{\textcolor{white}{2}} & \cellcolor[HTML]{445ca9}{\textcolor{white}{4.87}} & \cellcolor[HTML]{445ca9}{\textcolor{white}{\$342,500}} & \cellcolor[HTML]{445ca9}{\textcolor{white}{0}} & \cellcolor[HTML]{445ca9}{\textcolor{white}{60}} & \cellcolor[HTML]{445ca9}{\textcolor{white}{77.6}}\\
\cellcolor[HTML]{8baeaa}{St Patricks Anglo Vietnamese Church} & \cellcolor[HTML]{8baeaa}{1} & \cellcolor[HTML]{8baeaa}{5.64} & \cellcolor[HTML]{8baeaa}{\$1,197,000} & \cellcolor[HTML]{8baeaa}{21} & \cellcolor[HTML]{8baeaa}{53} & \cellcolor[HTML]{8baeaa}{67.0}\\
\cellcolor[HTML]{8baeaa}{Christ Ascension Church} & \cellcolor[HTML]{8baeaa}{1} & \cellcolor[HTML]{8baeaa}{1.47} & \cellcolor[HTML]{8baeaa}{\$414,000} & \cellcolor[HTML]{8baeaa}{18} & \cellcolor[HTML]{8baeaa}{49} & \cellcolor[HTML]{8baeaa}{64.8}\\
\cellcolor[HTML]{8baeaa}{Good Shepherd-of-the-Hills} & \cellcolor[HTML]{8baeaa}{1} & \cellcolor[HTML]{8baeaa}{3.12} & \cellcolor[HTML]{8baeaa}{\$117,600} & \cellcolor[HTML]{8baeaa}{0} & \cellcolor[HTML]{8baeaa}{16} & \cellcolor[HTML]{8baeaa}{63.2}\\
\cellcolor[HTML]{8baeaa}{St Johns Church} & \cellcolor[HTML]{8baeaa}{2} & \cellcolor[HTML]{8baeaa}{0.98} & \cellcolor[HTML]{8baeaa}{\$11,000} & \cellcolor[HTML]{8baeaa}{18} & \cellcolor[HTML]{8baeaa}{22} & \cellcolor[HTML]{8baeaa}{60.0}\\
\addlinespace
\cellcolor[HTML]{e9ab3f}{Grace Church Emmanuel Parish} & \cellcolor[HTML]{e9ab3f}{2} & \cellcolor[HTML]{e9ab3f}{6.41} & \cellcolor[HTML]{e9ab3f}{\$319,800} & \cellcolor[HTML]{e9ab3f}{14} & \cellcolor[HTML]{e9ab3f}{38} & \cellcolor[HTML]{e9ab3f}{58.2}\\
\cellcolor[HTML]{e9ab3f}{St Georges Church} & \cellcolor[HTML]{e9ab3f}{1} & \cellcolor[HTML]{e9ab3f}{1.83} & \cellcolor[HTML]{e9ab3f}{\$70,000} & \cellcolor[HTML]{e9ab3f}{0} & \cellcolor[HTML]{e9ab3f}{7} & \cellcolor[HTML]{e9ab3f}{57.0}\\
\bottomrule
\end{longtabu}

Looking at top 10 developent parcels and their congregations reveals a
striking pattern: all 7 congregations report average attendance at or
below 20 people, with a median attendance of just 14 people. This
universally low attendance suggests these properties represent not just
real estate opportunities, but potential candidates for congregational
transition---whether through relocation, merger, or strategic closure.

\subsection{Transition Opportunities}\label{transition-opportunities}

St.~Pauls on the Hill Church in Winchester (Frederick County)
exemplifies this dynamic. Despite maintaining 60 members on the rolls,
the congregation reports zero average attendance. With a development
score of 77.6---the highest in the Diocese---and nearly 5 acres of land,
St.~Pauls represents a flagship opportunity where VEREP engagement could
explore whether this congregation is ready for a supported transition
that unlocks significant development potential in a priority market.

Similarly, Good Shepherd-of-the-Hills in Albemarle County (63.2) and
St.~Georges Church in Page County (57.0) both report zero average
attendance despite having members on record. These congregations may
already be functionally inactive, making them strong candidates for
early VEREP outreach to clarify their status and discuss path forward
options.

\subsection{Merging Congregations}\label{merging-congregations}

Fairfax County presents a unique opportunity for coordinated engagement.
Three congregations---St.~Patricks Anglo Vietnamese Church (21
attendance), Holy Cross Korean Episcopal Church (25 attendance), and All
Saints Sharon Chapel (28 attendance)---are all operating with modest but
active congregations. Given their geographic proximity and similar
scale, VEREP could explore whether a merger strategy might strengthen
congregational life while freeing valuable property for development.

\subsection{Consolidation
Opportunities}\label{consolidation-opportunities}

Grace Church Emmanuel Parish in Fauquier County holds the largest
parcels among top developable congregations but averages only 14 in
attendance with 38 members. This table only reflects the two highest
scored of their parcels, or 6.4 acres out of their 14.7 total acres
across 3 parcels. As a multi-parcel congregation, Grace Church may be
able to consolidating its footprint, potentially releasing surplus land
for development while maintaining a right-sized worship presence.

\begin{tcolorbox}[enhanced jigsaw, breakable, bottomrule=.15mm, title=\textcolor{quarto-callout-caution-color}{\faFire}\hspace{0.5em}{Caution}, toprule=.15mm, opacitybacktitle=0.6, toptitle=1mm, left=2mm, bottomtitle=1mm, coltitle=black, titlerule=0mm, colbacktitle=quarto-callout-caution-color!10!white, colback=white, leftrule=.75mm, arc=.35mm, colframe=quarto-callout-caution-color-frame, rightrule=.15mm, opacityback=0]

Congregation data also has its flaws.

Aquia Church in Stafford County, for example, claims 1,136 members but
zero average attendance across 5 parcels. This discrepancy warrants
VEREP engagement to understand the congregation's actual status and
whether the membership figure reflects historical rolls rather than
current participation.

\end{tcolorbox}

\bookmarksetup{startatroot}

\chapter{Property Profiles}\label{property-profiles}

\section{Overview}\label{overview}

The following section provides detailed profiles for each of the top 15
development opportunities. Each profile includes:

\begin{itemize}
\tightlist
\item
  Site characteristics and current use
\item
  Congregation context (where applicable)
\item
  Development opportunities and strategies
\item
  Key considerations and next steps
\end{itemize}

All of these priority sites would benefit from further analysis of
housing needs in the area to determine best development fit.

\section*{1. St Pauls on the Hill
Church}\label{st-pauls-on-the-hill-church}
\addcontentsline{toc}{section}{1. St Pauls on the Hill Church}

\markright{1. St Pauls on the Hill Church}

\textbf{📍 WINCHESTER, FREDERICK County}

{Dev Score: 77.6} {2 Parcels}

\subsection{Overview}

{4.87}

{Total Acres}

{\$342K}

{Assessed Value}

{0}

{Avg Attendance}

{6/20}

{Walkability}

\begin{tcolorbox}[enhanced jigsaw, breakable, bottomrule=.15mm, title=\textcolor{quarto-callout-tip-color}{\faLightbulb}\hspace{0.5em}{📋 Multi-Parcel Opportunity}, toprule=.15mm, opacitybacktitle=0.6, toptitle=1mm, left=2mm, bottomtitle=1mm, coltitle=black, titlerule=0mm, colbacktitle=quarto-callout-tip-color!10!white, colback=white, leftrule=.75mm, arc=.35mm, colframe=quarto-callout-tip-color-frame, rightrule=.15mm, opacityback=0]

This congregation owns \textbf{2 parcels} totaling \textbf{4.87 acres},
which may present consolidation or phased development opportunities.

\end{tcolorbox}

\textbf{⛪ Congregation:} St Pauls on the Hill Church

\textbf{👥 Members:} 60

\textbf{💰 Avg Pledge:} \$0

\subsection{Congregational Health}

\subsubsection{Attendance Status}\label{attendance-status}

\textbf{⚠️ Zero Average Attendance}

This congregation reports no regular attendance despite having 60
members on the rolls. This may indicate:

\begin{itemize}
\tightlist
\item
  Functional inactivity requiring status clarification
\item
  Outdated membership records
\item
  A congregation ready for supported transition
\end{itemize}

\subsubsection{Financial Trend}\label{financial-trend}

\textbf{💰 Significant Decline}

Plate \& pledge has decreased by \textbf{100\%} over the past decade.

\subsection{Property Details}

\begin{longtable}[]{@{}ll@{}}
\toprule\noalign{}
Metric & Value \\
\midrule\noalign{}
\endhead
\bottomrule\noalign{}
\endlastfoot
Total Parcels & 2 \\
Total Acreage & 4.87 acres \\
Total Assessed Value & \$342,500 \\
Mean Dev Score & 77.6 \\
Max Dev Score & 79.2 \\
Walkability Score & 6/20 \\
Location & WINCHESTER, FREDERICK \\
\end{longtable}

\textbf{Parcel ID(s):}
\texttt{54\ \ \ \ A\ \ \ 128,\ 54\ \ \ \ A\ \ \ 128A}

\textbf{Addresses:}

\begin{itemize}
\tightlist
\item
  1527 SENSENY RD, WINCHESTER 22602
\item
  WINCHESTER 22602
\end{itemize}

\subsection{VEREP Action}

\textbf{🎯 Why Engage:} Zero attendance suggests potential for
congregational transition

\textbf{✅ Recommended Action:} Assess congregational status and
readiness for transition

\subsubsection{Suggested Next Steps}\label{suggested-next-steps}

\begin{enumerate}
\def\labelenumi{\arabic{enumi}.}
\tightlist
\item
  Contact diocesan staff to verify congregation's current status
\item
  Schedule site visit to assess property condition
\item
  If inactive, begin transition planning process
\item
  Engage remaining members on legacy preservation options
\end{enumerate}

\subsubsection{Key Contacts to Identify}\label{key-contacts-to-identify}

\begin{itemize}
\tightlist
\item[$\square$]
  Vestry leadership / Senior Warden
\item[$\square$]
  Diocesan regional representative
\item[$\square$]
  Property committee chair (if applicable)
\end{itemize}

\begin{center}\rule{0.5\linewidth}{0.5pt}\end{center}

\section*{2. St Patricks Anglo Vietnamese
Church}\label{st-patricks-anglo-vietnamese-church}
\addcontentsline{toc}{section}{2. St Patricks Anglo Vietnamese Church}

\markright{2. St Patricks Anglo Vietnamese Church}

\textbf{📍 FALLS CHURCH, FAIRFAX County}

{Dev Score: 67} {1 Parcel}

\subsection{Overview}

{5.64}

{Total Acres}

{\$1,197K}

{Assessed Value}

{21}

{Avg Attendance}

{13/20}

{Walkability}

\textbf{⛪ Congregation:} St Patricks Anglo Vietnamese Church

\textbf{👥 Members:} 53

\textbf{💰 Avg Pledge:} \$45,398

\begin{tcolorbox}[enhanced jigsaw, breakable, bottomrule=.15mm, title=\textcolor{quarto-callout-note-color}{\faInfo}\hspace{0.5em}{🚶 Moderate Walkability}, toprule=.15mm, opacitybacktitle=0.6, toptitle=1mm, left=2mm, bottomtitle=1mm, coltitle=black, titlerule=0mm, colbacktitle=quarto-callout-note-color!10!white, colback=white, leftrule=.75mm, arc=.35mm, colframe=quarto-callout-note-color-frame, rightrule=.15mm, opacityback=0]

This site scores \textbf{13/20} on walkability. Some transit access may
support density considerations.

\end{tcolorbox}

\subsection{Congregational Health}

\subsubsection{Attendance Status}\label{attendance-status-1}

\textbf{✓ Active Congregation}

21 average attendees with 53 members.

\subsubsection{Financial Trend}\label{financial-trend-1}

\textbf{💰 Significant Decline}

Plate \& pledge has decreased by \textbf{55.2\%} over the past decade.

\subsection{Property Details}

\begin{longtable}[]{@{}ll@{}}
\toprule\noalign{}
Metric & Value \\
\midrule\noalign{}
\endhead
\bottomrule\noalign{}
\endlastfoot
Total Parcels & 1 \\
Total Acreage & 5.64 acres \\
Total Assessed Value & \$1,197,000 \\
Mean Dev Score & 67 \\
Walkability Score & 13/20 \\
Location & FALLS CHURCH, FAIRFAX \\
\end{longtable}

\textbf{Parcel ID(s):} \texttt{0601\ 01\ \ 0079}

\subsection{VEREP Action}

\textbf{🎯 Why Engage:} Large combined acreage enables significant
development opportunity

\textbf{✅ Recommended Action:} Engage leadership on development
partnership options

\subsubsection{Suggested Next Steps}\label{suggested-next-steps-1}

\begin{enumerate}
\def\labelenumi{\arabic{enumi}.}
\tightlist
\item
  Schedule introductory meeting with vestry leadership
\item
  Present development partnership models and case studies
\item
  Conduct preliminary site feasibility assessment
\item
  Develop customized proposal based on congregation's goals
\end{enumerate}

\subsubsection{Key Contacts to
Identify}\label{key-contacts-to-identify-1}

\begin{itemize}
\tightlist
\item[$\square$]
  Vestry leadership / Senior Warden
\item[$\square$]
  Diocesan regional representative
\item[$\square$]
  Property committee chair (if applicable)
\end{itemize}

\begin{center}\rule{0.5\linewidth}{0.5pt}\end{center}

\section*{3. Christ Ascension Church}\label{christ-ascension-church}
\addcontentsline{toc}{section}{3. Christ Ascension Church}

\markright{3. Christ Ascension Church}

\textbf{📍 RICHMOND, RICHMOND-CITY County}

{Dev Score: 64.8} {1 Parcel}

\subsection{Overview}

{1.47}

{Total Acres}

{\$414K}

{Assessed Value}

{18}

{Avg Attendance}

{12.3/20}

{Walkability}

\textbf{⛪ Congregation:} Christ Ascension Church

\textbf{👥 Members:} 49

\textbf{💰 Avg Pledge:} \$29,472

\begin{tcolorbox}[enhanced jigsaw, breakable, bottomrule=.15mm, title=\textcolor{quarto-callout-note-color}{\faInfo}\hspace{0.5em}{🚶 Moderate Walkability}, toprule=.15mm, opacitybacktitle=0.6, toptitle=1mm, left=2mm, bottomtitle=1mm, coltitle=black, titlerule=0mm, colbacktitle=quarto-callout-note-color!10!white, colback=white, leftrule=.75mm, arc=.35mm, colframe=quarto-callout-note-color-frame, rightrule=.15mm, opacityback=0]

This site scores \textbf{12.3/20} on walkability. Some transit access
may support density considerations.

\end{tcolorbox}

\subsection{Congregational Health}

\subsubsection{Attendance Status}\label{attendance-status-2}

\textbf{📉 Low Attendance}

With only 18 average attendees and 49 members, this congregation is
operating at reduced capacity.

\subsubsection{Financial Trend}\label{financial-trend-2}

\textbf{💰 Significant Decline}

Plate \& pledge has decreased by \textbf{69.4\%} over the past decade.

\subsection{Property Details}

\begin{longtable}[]{@{}ll@{}}
\toprule\noalign{}
Metric & Value \\
\midrule\noalign{}
\endhead
\bottomrule\noalign{}
\endlastfoot
Total Parcels & 1 \\
Total Acreage & 1.47 acres \\
Total Assessed Value & \$414,000 \\
Mean Dev Score & 64.8 \\
Walkability Score & 12.3/20 \\
Location & RICHMOND, RICHMOND-CITY \\
\end{longtable}

\textbf{Parcel ID(s):} \texttt{N0170498039}

\subsection{VEREP Action}

\textbf{🎯 Why Engage:} Low attendance congregation may be open to
development partnership

\textbf{✅ Recommended Action:} Engage leadership on development
partnership options

\subsubsection{Suggested Next Steps}\label{suggested-next-steps-2}

\begin{enumerate}
\def\labelenumi{\arabic{enumi}.}
\tightlist
\item
  Schedule introductory meeting with vestry leadership
\item
  Present development partnership models and case studies
\item
  Conduct preliminary site feasibility assessment
\item
  Develop customized proposal based on congregation's goals
\end{enumerate}

\subsubsection{Key Contacts to
Identify}\label{key-contacts-to-identify-2}

\begin{itemize}
\tightlist
\item[$\square$]
  Vestry leadership / Senior Warden
\item[$\square$]
  Diocesan regional representative
\item[$\square$]
  Property committee chair (if applicable)
\end{itemize}

\begin{center}\rule{0.5\linewidth}{0.5pt}\end{center}

\section*{4. Good
Shepherd-of-the-Hills}\label{good-shepherd-of-the-hills}
\addcontentsline{toc}{section}{4. Good Shepherd-of-the-Hills}

\markright{4. Good Shepherd-of-the-Hills}

\textbf{📍 FREE UNION, ALBEMARLE County}

{Dev Score: 63.2} {1 Parcel}

\subsection{Overview}

{3.12}

{Total Acres}

{\$118K}

{Assessed Value}

{0}

{Avg Attendance}

{6.2/20}

{Walkability}

\textbf{⛪ Congregation:} Good Shepherd-of-the-Hills

\textbf{👥 Members:} 16

\textbf{💰 Avg Pledge:} \$0

\subsection{Congregational Health}

\subsubsection{Attendance Status}\label{attendance-status-3}

\textbf{⚠️ Zero Average Attendance}

This congregation reports no regular attendance despite having 16
members on the rolls. This may indicate:

\begin{itemize}
\tightlist
\item
  Functional inactivity requiring status clarification
\item
  Outdated membership records
\item
  A congregation ready for supported transition
\end{itemize}

\subsubsection{Financial Trend}\label{financial-trend-3}

\textbf{💰 Significant Decline}

Plate \& pledge has decreased by \textbf{100\%} over the past decade.

\subsection{Property Details}

\begin{longtable}[]{@{}ll@{}}
\toprule\noalign{}
Metric & Value \\
\midrule\noalign{}
\endhead
\bottomrule\noalign{}
\endlastfoot
Total Parcels & 1 \\
Total Acreage & 3.12 acres \\
Total Assessed Value & \$117,600 \\
Mean Dev Score & 63.2 \\
Walkability Score & 6.2/20 \\
Location & FREE UNION, ALBEMARLE \\
\end{longtable}

\textbf{Parcel ID(s):} \texttt{007000000002A0}

\subsection{VEREP Action}

\textbf{🎯 Why Engage:} Zero attendance suggests potential for
congregational transition

\textbf{✅ Recommended Action:} Assess congregational status and
readiness for transition

\subsubsection{Suggested Next Steps}\label{suggested-next-steps-3}

\begin{enumerate}
\def\labelenumi{\arabic{enumi}.}
\tightlist
\item
  Contact diocesan staff to verify congregation's current status
\item
  Schedule site visit to assess property condition
\item
  If inactive, begin transition planning process
\item
  Engage remaining members on legacy preservation options
\end{enumerate}

\subsubsection{Key Contacts to
Identify}\label{key-contacts-to-identify-3}

\begin{itemize}
\tightlist
\item[$\square$]
  Vestry leadership / Senior Warden
\item[$\square$]
  Diocesan regional representative
\item[$\square$]
  Property committee chair (if applicable)
\end{itemize}

\begin{center}\rule{0.5\linewidth}{0.5pt}\end{center}

\section*{5. St Johns Church}\label{st-johns-church}
\addcontentsline{toc}{section}{5. St Johns Church}

\markright{5. St Johns Church}

\textbf{📍 COLUMBIA, FLUVANNA County}

{Dev Score: 60} {2 Parcels}

\subsection{Overview}

{0.98}

{Total Acres}

{\$11K}

{Assessed Value}

{18}

{Avg Attendance}

{3/20}

{Walkability}

\begin{tcolorbox}[enhanced jigsaw, breakable, bottomrule=.15mm, title=\textcolor{quarto-callout-tip-color}{\faLightbulb}\hspace{0.5em}{📋 Multi-Parcel Opportunity}, toprule=.15mm, opacitybacktitle=0.6, toptitle=1mm, left=2mm, bottomtitle=1mm, coltitle=black, titlerule=0mm, colbacktitle=quarto-callout-tip-color!10!white, colback=white, leftrule=.75mm, arc=.35mm, colframe=quarto-callout-tip-color-frame, rightrule=.15mm, opacityback=0]

This congregation owns \textbf{2 parcels} totaling \textbf{0.98 acres},
which may present consolidation or phased development opportunities.

\end{tcolorbox}

\textbf{⛪ Congregation:} St Johns Church

\textbf{👥 Members:} 22

\textbf{💰 Avg Pledge:} \$25,987

\subsection{Congregational Health}

\subsubsection{Attendance Status}\label{attendance-status-4}

\textbf{📉 Low Attendance}

With only 18 average attendees and 22 members, this congregation is
operating at reduced capacity.

\subsubsection{Financial Trend}\label{financial-trend-4}

\textbf{💰 Modest Decline}

Plate \& pledge has decreased by \textbf{11\%} over the past decade.

\subsection{Property Details}

\begin{longtable}[]{@{}ll@{}}
\toprule\noalign{}
Metric & Value \\
\midrule\noalign{}
\endhead
\bottomrule\noalign{}
\endlastfoot
Total Parcels & 2 \\
Total Acreage & 0.98 acres \\
Total Assessed Value & \$11,000 \\
Mean Dev Score & 60 \\
Max Dev Score & 62 \\
Walkability Score & 3/20 \\
Location & COLUMBIA, FLUVANNA \\
\end{longtable}

\textbf{Parcel ID(s):} \texttt{54A-1-27,\ 54A-1-11}

\textbf{Addresses:}

\begin{itemize}
\tightlist
\item
  64 K ST, COLUMBIA 23038
\item
  FAYETTE ST, COLUMBIA 23038
\end{itemize}

\subsection{VEREP Action}

\textbf{🎯 Why Engage:} Low attendance congregation may be open to
development partnership

\textbf{✅ Recommended Action:} Map all parcels to identify
consolidation or phased development approach

\subsubsection{Suggested Next Steps}\label{suggested-next-steps-4}

\begin{enumerate}
\def\labelenumi{\arabic{enumi}.}
\tightlist
\item
  Map all parcels to identify consolidation opportunities
\item
  Assess which parcel(s) best serve ongoing worship needs
\item
  Evaluate surplus parcels for development potential
\item
  Present phased transition plan to vestry
\end{enumerate}

\subsubsection{Key Contacts to
Identify}\label{key-contacts-to-identify-4}

\begin{itemize}
\tightlist
\item[$\square$]
  Vestry leadership / Senior Warden
\item[$\square$]
  Diocesan regional representative
\item[$\square$]
  Property committee chair (if applicable)
\end{itemize}

\begin{center}\rule{0.5\linewidth}{0.5pt}\end{center}

\section*{6. Grace Church Emmanuel
Parish}\label{grace-church-emmanuel-parish}
\addcontentsline{toc}{section}{6. Grace Church Emmanuel Parish}

\markright{6. Grace Church Emmanuel Parish}

\textbf{📍 MIDLAND, FAUQUIER County}

{Dev Score: 58.2} {2 Parcels}

\subsection{Overview}

{6.41}

{Total Acres}

{\$320K}

{Assessed Value}

{14}

{Avg Attendance}

{5.7/20}

{Walkability}

\begin{tcolorbox}[enhanced jigsaw, breakable, bottomrule=.15mm, title=\textcolor{quarto-callout-tip-color}{\faLightbulb}\hspace{0.5em}{📋 Multi-Parcel Opportunity}, toprule=.15mm, opacitybacktitle=0.6, toptitle=1mm, left=2mm, bottomtitle=1mm, coltitle=black, titlerule=0mm, colbacktitle=quarto-callout-tip-color!10!white, colback=white, leftrule=.75mm, arc=.35mm, colframe=quarto-callout-tip-color-frame, rightrule=.15mm, opacityback=0]

This congregation owns \textbf{2 parcels} totaling \textbf{6.41 acres},
which may present consolidation or phased development opportunities.

\end{tcolorbox}

\textbf{⛪ Congregation:} Grace Church Emmanuel Parish

\textbf{👥 Members:} 38

\textbf{💰 Avg Pledge:} \$72,564

\subsection{Congregational Health}

\subsubsection{Attendance Status}\label{attendance-status-5}

\textbf{📉 Low Attendance}

With only 14 average attendees and 38 members, this congregation is
operating at reduced capacity.

\subsubsection{Financial Trend}\label{financial-trend-5}

\textbf{💰 Modest Decline}

Plate \& pledge has decreased by \textbf{18.1\%} over the past decade.

\subsection{Property Details}

\begin{longtable}[]{@{}ll@{}}
\toprule\noalign{}
Metric & Value \\
\midrule\noalign{}
\endhead
\bottomrule\noalign{}
\endlastfoot
Total Parcels & 2 \\
Total Acreage & 6.41 acres \\
Total Assessed Value & \$319,800 \\
Mean Dev Score & 58.2 \\
Max Dev Score & 58.2 \\
Walkability Score & 5.7/20 \\
Location & MIDLAND, FAUQUIER \\
\end{longtable}

\textbf{Parcel ID(s):} \texttt{7901-78-0662-000,\ 7901-79-2032-000}

\textbf{Addresses:}

\begin{itemize}
\tightlist
\item
  5096 GRACE CHURCH LN, MIDLAND 22728
\item
  5094 GRACE CHURCH LN, MIDLAND 22728
\end{itemize}

\subsection{VEREP Action}

\textbf{🎯 Why Engage:} Low attendance congregation may be open to
development partnership

\textbf{✅ Recommended Action:} Map all parcels to identify
consolidation or phased development approach

\subsubsection{Suggested Next Steps}\label{suggested-next-steps-5}

\begin{enumerate}
\def\labelenumi{\arabic{enumi}.}
\tightlist
\item
  Map all parcels to identify consolidation opportunities
\item
  Assess which parcel(s) best serve ongoing worship needs
\item
  Evaluate surplus parcels for development potential
\item
  Present phased transition plan to vestry
\end{enumerate}

\subsubsection{Key Contacts to
Identify}\label{key-contacts-to-identify-5}

\begin{itemize}
\tightlist
\item[$\square$]
  Vestry leadership / Senior Warden
\item[$\square$]
  Diocesan regional representative
\item[$\square$]
  Property committee chair (if applicable)
\end{itemize}

\begin{center}\rule{0.5\linewidth}{0.5pt}\end{center}

\section*{7. St Georges Church}\label{st-georges-church}
\addcontentsline{toc}{section}{7. St Georges Church}

\markright{7. St Georges Church}

\textbf{📍 STANLEY, PAGE County}

{Dev Score: 57} {1 Parcel}

\subsection{Overview}

{1.83}

{Total Acres}

{\$70K}

{Assessed Value}

{0}

{Avg Attendance}

{4.5/20}

{Walkability}

\textbf{⛪ Congregation:} St Georges Church

\textbf{👥 Members:} 7

\textbf{💰 Avg Pledge:} \$0

\subsection{Congregational Health}

\subsubsection{Attendance Status}\label{attendance-status-6}

\textbf{⚠️ Zero Average Attendance}

This congregation reports no regular attendance despite having 7 members
on the rolls. This may indicate:

\begin{itemize}
\tightlist
\item
  Functional inactivity requiring status clarification
\item
  Outdated membership records
\item
  A congregation ready for supported transition
\end{itemize}

\subsubsection{Financial Trend}\label{financial-trend-6}

\textbf{💰 Significant Decline}

Plate \& pledge has decreased by \textbf{100\%} over the past decade.

\subsection{Property Details}

\begin{longtable}[]{@{}ll@{}}
\toprule\noalign{}
Metric & Value \\
\midrule\noalign{}
\endhead
\bottomrule\noalign{}
\endlastfoot
Total Parcels & 1 \\
Total Acreage & 1.83 acres \\
Total Assessed Value & \$70,000 \\
Mean Dev Score & 57 \\
Walkability Score & 4.5/20 \\
Location & STANLEY, PAGE \\
\end{longtable}

\textbf{Parcel ID(s):} \texttt{91\ \ \ A\ \ \ \ \ \ \ 63}

\subsection{VEREP Action}

\textbf{🎯 Why Engage:} Zero attendance suggests potential for
congregational transition

\textbf{✅ Recommended Action:} Assess congregational status and
readiness for transition

\subsubsection{Suggested Next Steps}\label{suggested-next-steps-6}

\begin{enumerate}
\def\labelenumi{\arabic{enumi}.}
\tightlist
\item
  Contact diocesan staff to verify congregation's current status
\item
  Schedule site visit to assess property condition
\item
  If inactive, begin transition planning process
\item
  Engage remaining members on legacy preservation options
\end{enumerate}

\subsubsection{Key Contacts to
Identify}\label{key-contacts-to-identify-6}

\begin{itemize}
\tightlist
\item[$\square$]
  Vestry leadership / Senior Warden
\item[$\square$]
  Diocesan regional representative
\item[$\square$]
  Property committee chair (if applicable)
\end{itemize}

\begin{center}\rule{0.5\linewidth}{0.5pt}\end{center}

\part{Appendices}

\chapter*{Appendices}\label{appendices-1}
\addcontentsline{toc}{chapter}{Appendices}

\markboth{Appendices}{Appendices}

\begin{tcolorbox}[enhanced jigsaw, breakable, bottomrule=.15mm, title=\textcolor{quarto-callout-note-color}{\faInfo}\hspace{0.5em}{Technical Methodology}, toprule=.15mm, opacitybacktitle=0.6, toptitle=1mm, left=2mm, bottomtitle=1mm, coltitle=black, titlerule=0mm, colbacktitle=quarto-callout-note-color!10!white, colback=white, leftrule=.75mm, arc=.35mm, colframe=quarto-callout-note-color-frame, rightrule=.15mm, opacityback=0]

\textbf{Scoring Algorithm Details}

Each property receives scores across six dimensions (0-100 scale), which
are then weighted and combined:

\begin{enumerate}
\def\labelenumi{\arabic{enumi}.}
\tightlist
\item
  \textbf{Size Score (20\% weight):} Properties between 0.5-5 acres
  score highest (100 points)
\item
  \textbf{Use Score (25\% weight):} Parking lots (95) and open space
  (90) score highest
\item
  \textbf{Location Score (20\% weight):} Based on walkability index and
  transit access
\item
  \textbf{Financial Score (15\% weight):} Congregation pledge trends
  indicate development need
\item
  \textbf{Market Score (10\% weight):} Area median income and LIHTC
  eligibility
\item
  \textbf{Zoning Score (10\% weight):} Development-friendly zoning
  receives 80 points
\end{enumerate}

\textbf{Constraint penalties} are applied after the weighted average,
with major penalties for: - Flood zones: -40 points - Significant
wetlands (\textgreater25\%): -35 points\\
- Easements: -30 points - Historic districts: -25 points

Final scores are capped between 0-100.

\end{tcolorbox}

\begin{tcolorbox}[enhanced jigsaw, breakable, bottomrule=.15mm, title=\textcolor{quarto-callout-note-color}{\faInfo}\hspace{0.5em}{Data Processing Notes}, toprule=.15mm, opacitybacktitle=0.6, toptitle=1mm, left=2mm, bottomtitle=1mm, coltitle=black, titlerule=0mm, colbacktitle=quarto-callout-note-color!10!white, colback=white, leftrule=.75mm, arc=.35mm, colframe=quarto-callout-note-color-frame, rightrule=.15mm, opacityback=0]

\textbf{Congregation-Parcel Joining:} Properties were matched to
congregation data using the \texttt{congr\_name} field. Properties
without matches received neutral financial scores.

\textbf{Missing Data Handling:} - Missing walkability scores defaulted
to 0 - Missing income data received median market scores (50) -
Properties without area or coordinates were excluded

\textbf{Quality Filters:} - Minimum parcel size: 0.1 acres - Required
valid lat/lon coordinates - Valid building years: 1700-2025

\end{tcolorbox}




\end{document}
